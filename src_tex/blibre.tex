
\documentclass[twoside,12pt]{book}
\usepackage{polyglossia}
\usepackage{fontspec}
\usepackage{xunicode}
\usepackage{fancyhdr}
\usepackage{lettrine}
\usepackage{microtype}
\usepackage{color}
\usepackage{xcolor}
\usepackage{geometry}
%\usepackage[showframe]{geometry}
%\usepackage[para,symbol*,bottom,flushmargin]{footmisc}
%\usepackage[para,symbol*,bottom]{footmisc}
%\usepackage[para,symbol*]{footmisc}
\usepackage[para,symbol*,flushmargin]{footmisc}
\usepackage{fourier-orns}
\usepackage{graphicx}
\usepackage{booktabs}
\usepackage{multicol}
\usepackage[object=vectorian]{pgfornament}
\usepackage[explicit]{titlesec}

\renewcommand{\familydefault}{pplj} 
\setdefaultlanguage{english}
\setotherlanguage{hebrew}

%\renewcommand{\familydefault}{\sfdefault}
\renewcommand{\familydefault}{\rmdefault}

%\titlespacing{\paragraph}{0pt}{0pt}{.5em}[]
\titlespacing{\paragraph}{0pt}{0pt}{.1em}[]

% \raggedbottom

% Footnote settings
%----------------------------------------------------

% The prefix 'bl' is used for all defined vars, commands, etc. meaning 'biblia-libre'

% Using the same symbol for all footnotes because:
%	1. many footnotes (like in a bible) always crash footnote counter.
%	2. footnote counter is reset every footnote.
%	3. footnotes must not use numbers to avoid confusing them with verse numbers.

\DefineFNsymbols*{blSymbols}{*****************************************************************
**********************************************************************************************}

\setfnsymbol{blSymbols}

% \interfootnotelinepenalty=0
% \addtolength{\skip\footins}{2pc plus 5pt}
% \setlength{\skip\footins}{5pt}

% Page layout.
%-----------------------------------------------------

\geometry{
	letterpaper,
	top = 20mm,
	bottom = 20mm,
	inner = 25mm,
	outer = 10mm,
	footskip = 8mm,
	headsep = 3mm,
	headheight = 20mm,
}

%\pagenumbering{gobble}

% Fonts to use. 
%-----------------------------------------------------

% in fonts folder
\newfontface\blSetDSHermannGotisch{DSHermannGotisch}
\newfontface\blSetDSRomantiques{DSRomantiques}
\newfontface\blSetWieynkFrakturVignetten{WieynkFrakturVignetten}
\newfontface\blSetRothenburgDecorative{RothenburgDecorative}

% in texlive full install
\input Acorn.fd
\newfontface\blSetAcorn{Acorn}

% in system (linux)
\newfontface\blSetDavidCLM{David CLM}

\newcommand{\blHebrewOne}[2]{\blSetDavidCLM\fontsize{#1}{40}\selectfont \RL{#2}}

% Some colors to use.
%-----------------------------------------------------
\definecolor{blMarron}{RGB}{60,30,10}
\definecolor{blDarkblue}{RGB}{0,0,80}
\definecolor{blLightblue}{RGB}{80,80,80}
\definecolor{blDarkgreen}{RGB}{0,80,0}
\definecolor{blDarkgray}{RGB}{0, 51, 153}
\definecolor{blDarkred}{RGB}{80,0,0}

\definecolor{blLayoutColor}{RGB}{0, 51, 153}
% \definecolor{blLayoutColor}{blue}{0.6}
\colorlet{blNormalColor}{black}

% Header and Footer variables.
%-----------------------------------------------------

\newcommand{\blBookNameMeaning}{}
\newcommand{\blBookNumber}{MyBookNumber}
\newcommand{\blHebrewBookName}{MyHebrewBookName}
\newcommand{\blBookName}{MyBookName}
\newcommand{\blBookNameTranslit}{MyBookNameTranslit}

\newcommand{\blChapterNumber}{MyChapterNum}

% Variable setters y formated getters
%-----------------------------------------------------

\newcommand{\blSetBookNameMeaning}[1]{\renewcommand{\blBookNameMeaning}{#1}}
\newcommand{\blSetBookNumber}[1]{\renewcommand{\blBookNumber}{#1}}
\newcommand{\blSetHebrewBookName}[1]{\renewcommand{\blHebrewBookName}{#1}}
\newcommand{\blSetBookName}[1]{\renewcommand{\blBookName}{#1}}
\newcommand{\blSetBookNameTranslit}[1]{\renewcommand{\blBookNameTranslit}{#1}}

\newcommand{\blSetChapterNumber}[1]{\renewcommand{\blChapterNumber}{#1}}

\newcommand{\blGetVerse}[1]{%
\setcounter{footnote}{0}%
\textsuperscript{#1}%
}

\newcommand{\blGetVerseX}[2]{%
\footnote{#2}\blGetVerse{#1}%
}

\newcommand{\blIniChar}[1]{%
\blSetAcorn\selectfont{#1}%
}

\newcommand{\blIniWord}[1]{%
\blSetRothenburgDecorative\selectfont{#1}%
}

\newcommand{\blIniNum}[1]{%
\blSetDSRomantiques\selectfont{#1}%
}


% Header and Footer look.
%-----------------------------------------------------

\newcommand{\blOrnament}{ \vspace{3ex}\noindent \textcolor{blLayoutColor}{\hrulefill~ \raisebox{-2.5pt}[10pt][10pt]{\leafright \decofourleft \decothreeleft  \aldineright \decotwo \floweroneleft \decoone   \floweroneright \decotwo \aldineleft\decothreeright \decofourright \leafleft} ~  \hrulefill \\ \vspace{3ex}}}

\newcommand{\blOddOrnament}{\noindent \textcolor{blLayoutColor}{ \raisebox{-1.9pt}[10pt][10pt]{\leafright} \hrulefill \raisebox{-1.9pt}[10pt][10pt]{\leafright \decofourleft \decothreeleft  \aldineright \decotwo \floweroneleft \decoone}}}

\newcommand{\blEvenOrnament}{\noindent \textcolor{blLayoutColor}{\raisebox{-1.9pt}[10pt][10pt]{\decoone \floweroneright \decotwo \aldineleft \decothreeright \decofourright \leafleft} \hrulefill \raisebox{-1.9pt}[10pt][10pt]{\leafleft}}}

\newcommand{\blTopOrnament}{\noindent \textcolor{blLayoutColor}{\raisebox{-2.1pt}[10pt][10pt]{\leafright} \hrulefill \raisebox{-2.1pt}[10pt][10pt]{~~~\decofourleft \decotwo \decofourright~~~} \hrulefill \raisebox{-2.1pt}[10pt][10pt]{ \leafleft}} \vspace{0mm}}

\newcommand\blGetOrnament[3][black]{
	\resizebox{#2}{!}{\pgfornament[color = #1,ydelta=-40pt]{#3}}
}

\newcommand\blBookOrnament{{\blSetWieynkFrakturVignetten\fontsize{50}{40}\selectfont P}}

\newcommand\blGetBookHebrewTitle{\hfill \hfill {\blHebrewOne{50}{\blHebrewBookName}} \hfill \blBookOrnament}

% \blGetOrnament[blLayoutColor]{0.7\linewidth}{85} \\
% \blGetOrnament[blLayoutColor]{0.2\linewidth}{71} \\

\newcommand\blGetBookTitle{
	\centerline{\textcolor{blLayoutColor}{\blGetBookHebrewTitle}}
	\centerline{\textcolor{blLayoutColor}{\blBookNameTranslit}}
	\centerline{\textcolor{blLayoutColor}{\blBookNameMeaning}}
	\centerline{\textcolor{blLayoutColor}{(\blBookName)}}
}


\makeatletter
\def\headrule{\blTopOrnament}
\makeatother

\pagestyle{fancy}

\fancyhf{}

\newcommand{\blTopInner}{\textcolor{blLayoutColor}{\blBookName \space \blChapterNumber}}
\newcommand{\blTopCenter}{\textcolor{blLayoutColor}{\thepage}}
\newcommand{\blTopOuter}{\textcolor{blLayoutColor}{\blBookNameTranslit \space \blChapterNumber}}

\newcommand{\blBottomOuter}{\blChapterNumber \space \blHebrewOne{12}{\blHebrewBookName}}
\newcommand{\blBottomOdd}{
	\blEvenOrnament \\ \large \hfill \sffamily\bf \textcolor{blLayoutColor}{\leafNE ~~~ \blBottomOuter}
}
\newcommand{\blBottomEven}{
	\blOddOrnament \\ \large \sffamily\bf \textcolor{blLayoutColor}{\blBottomOuter ~~~ \reflectbox{\leafNE}} \hfill
}

\fancyhead[LO,RE]{\blTopInner}
\fancyhead[CO,CE]{\blTopCenter}
\fancyhead[RO,LE]{\blTopOuter}

\fancyfoot[LO]{\blBottomOdd}
\fancyfoot[RE]{\blBottomEven}

\fancypagestyle{plain}{
	\fancyhead{}
	\renewcommand{\headrulewidth}{0pt}
	\renewcommand{\headrule}{}
}

\newcommand{\blStartBook}{%
	\thispagestyle{plain}%
	\blGetBookTitle%
}

\newcommand{\blEndBook}{%
	\vspace{1em}
	\centerline{\textcolor{blLayoutColor}{\reflectbox{\blBookOrnament}\blBookOrnament}}%
	\cleardoublepage%
}

\newcommand{\blSubTitle}[1]{
	\vspace{1em}
	%\centerline{\textcolor{blLayoutColor}{\textbf{\fontsize{10}{3}\selectfont #1}}}%
	\centerline{\textcolor{blLayoutColor}{\textbf{\large #1 } \small}}
}


%====================================================================================================
%START OF DOCUMENT
%====================================================================================================
\begin{document}
%\begin{multicols}{2}

%\thispagestyle{empty}

\small

\blSetBookName{Génesis}
\blSetBookNameTranslit{Bereshit}
\blSetBookNameMeaning{En el principio}
\blSetHebrewBookName{בראשית}
\blSetBookNumber{1}
\blStartBook
% =================================================

\blSetChapterNumber{1}



\blSubTitle{La creación}

\lettrine{1}{}
\blGetVerseX{1}{Isa 48:6-Isa 48:7 Jer 31:22 Gen 1:27 Gen 5:1 Deu 4:32 Isa 45:12 Isa 43:1 Isa 43:15 Isa 65:18 Sal 51:10 Isa 65:17 Isa 66:22 Gen 14:22 Sal 124:8 Mat 28:18 Gen 2:9 }En el principio crió Dios los cielos y la tierra. 
\blGetVerseX{2}{Sal 104:6-Sal 104:9 Sal 18:15 Sal 24:2 Gen 10:9 Gen 13:10 }Y la tierra estaba desordenada y vacía, y las tinieblas estaban sobre la haz del abismo, y el Espíritu de Dios se movía sobre la haz de las aguas. 
\blGetVerseX{3}{Sal 33:6-Sal 33:9 Sal 148:5 Isa 48:13 Isa 55:10-Isa 55:11 Heb 11:3 2Co 4:6 3Ju 1:1-3Ju 1:4 3Ju 1:1-3Ju 1:4 3Ju 1:1-3Ju 1:4 }Y dijo Dios: Sea la luz: y fué la luz. 
\blGetVerse{4}Y vió Dios que la luz era buena: y apartó Dios la luz de las tinieblas. 
\blGetVerseX{5}{Sal 55:17 }Y llamó Dios á la luz Día, y á las tinieblas llamó Noche: y fué la tarde y la mañana un día. 
\blGetVerseX{6}{Gen 1:9 Gen 1:11 }Y dijo Dios: Haya expansión en medio de las aguas, y separe las aguas de las aguas. 
\blGetVerse{7}E hizo Dios la expansión, y apartó las aguas que estaban debajo de la expansión, de las aguas que estaban sobre la expansión: y fué así. 
\blGetVerseX{8}{Exo 20:4 Job 37:18 Gen 7:11 Sal 148:4 Isa 40:22 Gen 1:2 Sal 24:2 Sal 104:5 Sal 136:6 }Y llamó Dios á la expansión Cielos: y fué la tarde y la mañana el día segundo. 
\blGetVerseX{9}{Job 38:8-Job 38:11 Sal 104:6-Sal 104:9 Pro 8:27-Pro 8:30 Jer 5:22 Gen 1:2 }Y dijo Dios: Júntense las aguas que están debajo de los cielos en un lugar, y descúbrase la seca: y fué así. 
\blGetVerse{10}Y llamó Dios á la seca Tierra, y á la reunión de las aguas llamó Mares: y vió Dios que era bueno. 
\blGetVerseX{11}{Gen 1:28 }Y dijo Dios: Produzca la tierra hierba verde, hierba que dé simiente; árbol de fruto que dé fruto según su género, que su simiente esté en él, sobre la tierra: y fué así. 
\blGetVerse{12}Y produjo la tierra hierba verde, hierba que da simiente según su naturaleza, y árbol que da fruto, cuya simiente está en él, según su género: y vió Dios que era bueno. 
\blGetVerse{13}Y fué la tarde y la mañana el día tercero. 
\blGetVerseX{14}{Sal 81:3 }Y dijo Dios: Sean lumbreras en la expansión de los cielos para apartar el día y la noche: y sean por señales, y para las estaciones, y para días y años; 
\blGetVerse{15}Y sean por lumbreras en la expansión de los cielos para alumbrar sobre la tierra: y fue. 
\blGetVerse{16}E hizo Dios las dos grandes lumbreras; la lumbrera mayor para que señorease en el día, y la lumbrera menor para que señorease en la noche: hizo también las estrellas. 
\blGetVerse{17}Y púsolas Dios en la expansión de los cielos, para alumbrar sobre la tierra, 
\blGetVerseX{18}{Deu 4:19 Sof 1:5 Sal 8:3 Jer 31:35 }Y para señorear en el día y en la noche, y para apartar la luz y las tinieblas: y vió Dios que era bueno. 
\blGetVerse{19}Y fué la tarde y la mañana el día cuarto. 
\blGetVerse{20}Y dijo Dios: Produzcan las aguas reptil de ánima viviente, y aves que vuelen sobre la tierra, en la abierta expansión de los cielos. 
\blGetVerse{21}Y crió Dios las grandes ballenas, y toda cosa viva que anda arrastrando, que las aguas produjeron según su género, y toda ave alada según su especie: y vió Dios que era bueno. 
\blGetVerse{22}Y Dios los bendijo diciendo: Fructificad y multiplicad, y henchid las aguas en los mares, y las aves se multipliquen en la tierra. 
\blGetVerse{23}Y fué la tarde y la mañana el día quinto. 
\blGetVerse{24}Y dijo Dios: Produzca la tierra seres vivientes según su género, bestias y serpientes y animales de la tierra según su especie: y fué así. 
\blGetVerse{25}E hizo Dios animales de la tierra según su género, y ganado según su género, y todo animal que anda arrastrando sobre la tierra según su especie: y vió Dios que era bueno. 
\blGetVerseX{26}{Sal 8:4-Sal 8:8 }Y dijo Dios: Hagamos al hombre á nuestra imagen, conforme á nuestra semejanza; y señoree en los peces de la mar, y en las aves de los cielos, y en las bestias, y en toda la tierra, y en todo animal que anda arrastrando sobre la tierra. 
\blGetVerseX{27}{Gen 4:25 Gen 2:7 Gen 5:1 Gen 9:6 1Co 11:7 Stg 3:9 Mat 19:4 Mar 10:6 }Y crió Dios al hombre á su imagen, á imagen de Dios lo crió; varón y hembra los crió. 
\blGetVerseX{28}{Gen 5:1-Gen 5:2 Gen 17:16 Gen 17:20 Gen 22:17 Gen 26:12 Gen 26:24 Gen 28:3 Gen 49:22-Gen 49:26 Sal 128:0 }Y los bendijo Dios; y díjoles Dios: Fructificad y multiplicad, y henchid la tierra, y sojuzgadla, y señoread en los peces de la mar, y en las aves de los cielos, y en todas las bestias que se mueven sobre la tierra. 
\blGetVerse{29}Y dijo Dios: He aquí que os he dado toda hierba que da simiente, que está sobre la haz de toda la tierra; y todo árbol en que hay fruto de árbol que da simiente, seros ha para comer. 
\blGetVerseX{30}{Sal 50:9-Sal 50:13 }Y á toda bestia de la tierra, y á todas las aves de los cielos, y á todo lo que se mueve sobre la tierra, en que hay vida, toda hierba verde les será para comer: y fué así. 
\blGetVerse{31}Y vió Dios todo lo que había hecho, y he aquí que era bueno en gran manera. Y fué la tarde y la mañana el día sexto. 
% =================================================

\blSetChapterNumber{2}

\lettrine{2}{}
\blGetVerse{1}Y FUERON acabados los cielos y la tierra, y todo su ornamento. 
\blGetVerseX{2}{Exo 20:11 Exo 31:17 Heb 4:4 Heb 4:10 Jua 15:17-Jua 15:18 }Y acabó Dios en el día séptimo su obra que hizo, y reposó el día séptimo de toda su obra que había hecho. 
\blGetVerseX{3}{Gen 4:18 }Y bendijo Dios al día séptimo, y santificólo, porque en él reposó de toda su obra que había Dios criado y hecho. 


\blSubTitle{El hombre en el huerto del Edén}

\blGetVerseX{4}{Gen 1:1 Gen 5:1 Gen 1:1-Gen 2:4 Gen 4:26 Exo 3:14-Exo 3:15 }Estos son los orígenes de los cielos y de la tierra cuando fueron criados, el día que Jehová Dios hizo la tierra y los cielos, 
\blGetVerse{5}Y toda planta del campo antes que fuese en la tierra, y toda hierba del campo antes que naciese: porque aun no había Jehová Dios hecho llover sobre la tierra, ni había hombre para que labrase la tierra; 
\blGetVerse{6}Mas subía de la tierra un vapor, que regaba toda la faz de la tierra. 
\blGetVerseX{7}{Gen 3:19 Gen 1:27 Isa 45:9-Isa 45:11 Jer 18:1-Jer 18:6 Rom 9:21 1Co 15:45 }Formó, pues, Jehová Dios al hombre del polvo de la tierra, y alentó en su nariz soplo de vida; y fué el hombre en alma viviente. 
\blGetVerseX{8}{Isa 51:3 Eze 31:8-Eze 31:9 }Y había Jehová Dios plantado un huerto en Edén al oriente, y puso allí al hombre que había formado. 
\blGetVerseX{9}{Gen 3:22 Rev 2:7 Rev 22:2 Rev 22:14 Gen 1:1 Gen 3:22 }Y había Jehová Dios hecho nacer de la tierra todo árbol delicioso á la vista, y bueno para comer: también el árbol de vida en medio del huerto, y el árbol de ciencia del bien y del mal. 
\blGetVerse{10}Y salía de Edén un río para regar el huerto, y de allí se repartía en cuatro ramales. 
\blGetVerse{11}El nombre del uno era Pisón: éste es el que cerca toda la tierra de Havilah, donde hay oro: 
\blGetVerse{12}Y el oro de aquella tierra es bueno: hay allí también bdelio y piedra cornerina. 
\blGetVerse{13}El nombre del segundo río es Gihón: éste es el que rodea toda la tierra de Etiopía. 
\blGetVerse{14}Y el nombre del tercer río es Hiddekel: éste es el que va delante de Asiria. Y el cuarto río es el Eufrates. 
\blGetVerseX{15}{Gen 1:28 Gen 3:17-Gen 3:19 }Tomó, pues, Jehová Dios al hombre, y le puso en el huerto de Edén, para que lo labrara y lo guardase. 
\blGetVerse{16}Y mandó Jehová Dios al hombre, diciendo: De todo árbol del huerto comerás; 
\blGetVerseX{17}{Eze 28:2 Isa 14:13-Isa 14:14 Rom 6:23 }Mas del árbol de ciencia del bien y del mal no comerás de él; porque el día que de él comieres, morirás. 
\blGetVerse{18}Y dijo Jehová Dios: No es bueno que el hombre esté solo; haréle ayuda idónea para él. 
\blGetVerseX{19}{2Re 23:34 2Re 24:17 }Formó, pues, Jehová Dios de la tierra toda bestia del campo, y toda ave de los cielos, y trájolas á Adam, para que viese cómo les había de llamar; y todo lo que Adam llamó á los animales vivientes, ese es su nombre. 
\blGetVerse{20}Y puso Adam nombres á toda bestia y ave de los cielos y á todo animal del campo: mas para Adam no halló ayuda que estuviese idónea para él. 
\blGetVerse{21}Y Jehová Dios hizo caer sueño sobre Adam, y se quedó dormido: entonces tomó una de sus costillas, y cerró la carne en su lugar; 
\blGetVerse{22}Y de la costilla que Jehová Dios tomó del hombre, hizo una mujer, y trájola al hombre. 
\blGetVerse{23}Y dijo Adam: Esto es ahora hueso de mis huesos, y carne de mi carne: ésta será llamada Varona, porque del varón fué tomada. 
\blGetVerseX{24}{Mat 19:5 Mar 10:7-Mar 10:8 1Co 6:16 Efe 5:31 }Por tanto, dejará el hombre á su padre y á su madre, y allegarse ha á su mujer, y serán una sola carne. 
\blGetVerseX{25}{Eze 16:37 Ose 2:3 }Y estaban ambos desnudos, Adam y su mujer, y no se avergonzaban. 
% =================================================

\blSetChapterNumber{3}



\blSubTitle{Desobediencia del hombre}

\lettrine{3}{}
\blGetVerseX{1}{Mat 10:16 Jua 8:44 Rev 12:9 Rev 20:2 }EMPERO la serpiente era astuta, más que todos los animales del campo que Jehová Dios había hecho; la cual dijo á la mujer: ¿Conque Dios os ha dicho: No comáis de todo árbol del huerto? 
\blGetVerse{2}Y la mujer respondió á la serpiente: Del fruto de los árboles del huerto comemos; 
\blGetVerse{3}Mas del fruto del árbol que está en medio del huerto dijo Dios: No comeréis de él, ni le tocaréis, porque no muráis. 
\blGetVerse{4}Entonces la serpiente dijo á la mujer: No moriréis; 
\blGetVerse{5}Mas sabe Dios que el día que comiereis de él, serán abiertos vuestros ojos, y seréis como dioses sabiendo el bien y el mal. 
\blGetVerseX{6}{Rom 5:12 }Y vió la mujer que el árbol era bueno para comer, y que era agradable á los ojos, y árbol codiciable para alcanzar la sabiduría; y tomó de su fruto, y comió; y dió también á su marido, el cual comió así como ella. 
\blGetVerseX{7}{Gen 2:25 }Y fueron abiertos los ojos de entrambos, y conocieron que estaban desnudos: entonces cosieron hojas de higuera, y se hicieron delantales. 
\blGetVerse{8}Y oyeron la voz de Jehová Dios que se paseaba en el huerto al aire del día: y escondióse el hombre y su mujer de la presencia de Jehová Dios entre los árboles del huerto. 
\blGetVerseX{9}{Gen 4:9 }Y llamó Jehová Dios al hombre, y le dijo: ¿Dónde estás tú? 
\blGetVerse{10}Y él respondió: Oí tu voz en el huerto, y tuve miedo, porque estaba desnudo; y escondíme. 
\blGetVerse{11}Y díjole: ¿Quién te enseñó que estabas desnudo? ¿Has comido del árbol de que yo te mandé no comieses? 
\blGetVerse{12}Y el hombre respondió: La mujer que me diste por compañera me dió del árbol, y yo comí. 
\blGetVerseX{13}{2Co 11:3 1Ti 2:14 }Entonces Jehová Dios dijo á la mujer: ¿Qué es lo que has hecho? Y dijo la mujer: La serpiente me engañó, y comí. 
\blGetVerse{14}Y Jehová Dios dijo á la serpiente: Por cuanto esto hiciste, maldita serás entre todas las bestias y entre todos los animales del campo; sobre tu pecho andarás, y polvo comerás todos los días de tu vida: 
\blGetVerseX{15}{Rom 16:20 Rev 12:17 }Y enemistad pondré entre ti y la mujer, y entre tu simiente y la simiente suya; ésta te herirá en la cabeza, y tú le herirás en el calcañar. 
\blGetVerse{16}A la mujer dijo: Multiplicaré en gran manera tus dolores y tus preñeces; con dolor parirás los hijos; y á tu marido será tu deseo, y él se enseñoreará de ti. 
\blGetVerseX{17}{Gen 12:2-Gen 12:3 Gen 2:15 }Y al hombre dijo: Por cuanto obedeciste á la voz de tu mujer, y comiste del árbol de que te mandé diciendo, No comerás de él; maldita será la tierra por amor de ti; con dolor comerás de ella todos los días de tu vida; 
\blGetVerseX{18}{Rom 8:20 Heb 6:8 }Espinos y cardos te producirá, y comerás hierba del campo; 
\blGetVerse{19}En el sudor de tu rostro comerás el pan hasta que vuelvas á la tierra; porque de ella fuiste tomado: pues polvo eres, y al polvo serás tornado. 
\blGetVerse{20}Y llamó el hombre el nombre de su mujer, Eva; por cuanto ella era madre de todos lo vivientes. 
\blGetVerseX{21}{Gen 4:15 }Y Jehová Dios hizo al hombre y á su mujer túnicas de pieles, y vistiólos. 
\blGetVerseX{22}{Rev 22:14 }Y dijo Jehová Dios: He aquí el hombre es como uno de Nos sabiendo el bien y el mal: ahora, pues, porque no alargue su mano, y tome también del árbol de la vida, y coma, y viva para siempre: 
\blGetVerse{23}Y sacólo Jehová del huerto de Edén, para que labrase la tierra de que fué tomado. 
\blGetVerseX{24}{Exo 25:18 Gen 2:17 Rom 5:12 }Echó, pues, fuera al hombre, y puso al oriente del huerto de Edén querubines, y una espada encendida que se revolvía á todos lados, para guardar el camino del árbol de la vida. 
% =================================================

\blSetChapterNumber{4}



\blSubTitle{Caín y Abel}

\lettrine{4}{}
\blGetVerseX{1}{Gen 3:6 Jer 9:4 }Y CONOCIO Adam á su mujer Eva, la cual concibió y parió á Caín, y dijo: Adquirido he varón por Jehová. 
\blGetVerseX{2}{Jue 6:3-Jue 6:6 }Y después parió á su hermano Abel. Y fué Abel pastor de ovejas, y Caín fué labrador de la tierra. 
\blGetVerse{3}Y aconteció andando el tiempo, que Caín trajo del fruto de la tierra una ofrenda á Jehová. 
\blGetVerse{4}Y Abel trajo también de los primogénitos de sus ovejas, y de su grosura. Y miró Jehová con agrado á Abel y á su ofrenda; 
\blGetVerseX{5}{Gen 3:17 Exo 33:19 Deu 7:7-Deu 7:8 Rom 9:15 Heb 11:4 }Mas no miró propicio á Caín y á la ofrenda suya. Y ensañóse Caín en gran manera, y decayó su semblante. 
\blGetVerse{6}Entonces Jehová dijo á Caín: ¿Por qué te has ensañado, y por qué se ha inmutado tu rostro? 
\blGetVerseX{7}{1Pe 5:8 Gen 2:17 Deu 30:15-Deu 30:20 Gen 2:16-Gen 2:17 Gen 3:9 Gen 4:9 Gen 3:21 Gen 4:15 }Si bien hicieres, ¿no serás ensalzado? y si no hicieres bien, el pecado está á la puerta: con todo esto, á ti será su deseo, y tú te enseñorearás de él. 
\blGetVerseX{8}{Mat 23:35 Luc 11:51 1Ju 3:12 }Y habló Caín á su hermano Abel: y aconteció que estando ellos en el campo, Caín se levantó contra su hermano Abel, y le mató. 
\blGetVerseX{9}{Gen 3:9 Exo 32:22-Exo 32:24 }Y Jehová dijo á Caín: ¿Dónde está Abel tu hermano? Y él respondió: No sé; ¿soy yo guarda de mi hermano? 
\blGetVerseX{10}{Gen 3:13 Sal 9:12 Eze 24:7-Eze 24:8 Heb 12:24 }Y él le dijo: ¿Qué has hecho? La voz de la sangre de tu hermano clama á mí desde la tierra. 
\blGetVerse{11}Ahora pues, maldito seas tú de la tierra que abrió su boca para recibir la sangre de tu hermano de tu mano: 
\blGetVerse{12}Cuando labrares la tierra, no te volverá á dar su fuerza: errante y extranjero serás en la tierra. 
\blGetVerse{13}Y dijo Caín á Jehová: Grande es mi iniquidad para ser perdonada. 
\blGetVerse{14}He aquí me echas hoy de la faz de la tierra, y de tu presencia me esconderé; y seré errante y extranjero en la tierra; y sucederá que cualquiera que me hallare, me matará. 
\blGetVerseX{15}{Gen 4:23-Gen 4:24 Eze 9:4-Eze 9:6 Gen 3:21 }Y respondióle Jehová: Cierto que cualquiera que matare á Caín, siete veces será castigado. Entonces Jehová puso señal en Caín, para que no lo hiriese cualquiera que le hallara. 
\blGetVerse{16}Y salió Caín de delante de Jehová, y habitó en tierra de Nod, al oriente de Edén. 
\blGetVerseX{17}{Gen 11:10-Gen 11:26 Gen 12:4 Gen 26:4 Gen 28:14 Gen 4:8 }Y conoció Caín á su mujer, la cual concibió y parió á Henoch: y edificó una ciudad, y llamó el nombre de la ciudad del nombre de su hijo, Henoch. 
\blGetVerseX{18}{Gen 4:23-Gen 4:24 Sal 79:12 }Y á Henoch nació Irad, é Irad engendró á Mehujael, y Mehujael engendró á Methusael, y Methusael engendró á Lamech. 
\blGetVerse{19}Y tomó para sí Lamech dos mujeres; el nombre de la una fué Ada, y el nombre de la otra Zilla. 
\blGetVerse{20}Y Ada parió á Jabal, el cual fué padre de los que habitan en tiendas, y crían ganados. 


\blSetBookNameMeaning{En el comienzo}
\blSetBookNumber{1}
\blSetHebrewBookName{בראשית}
\blSetBookName{Genesis}
\blSetBookNameTranslit{Bereshit}
\blStartBook

\blSetChapterNumber{1}

%{\blSetAcorn\selectfont ABCDEF GHIJKL MNÑOPQ RSTUVW XYZ 01234 56789}

\lettrine{\blIniChar{FU}}{\blIniWord{turo}}
autem ex multis \blGetVerse{5}et per alia media concludit idem. \blGetVerse{3}Hic enim primo ostendit, quod anima \blGetVerseX{9}{my note 1 at the margin}rationalis 
non est virtus \blGetVerseX{1}{my note 2 at the margin}in corpore, ita quod secundum esse vel
operationem vel utrumque ad \blGetVerseX{7}{my note 3 at the margin}corporis harmoniam dependeat.

\lettrine{\blIniNum{150}}{Future}
autem ex multis \blGetVerse{5}et per alia media concludit idem. \blGetVerse{3}Hic enim primo ostendit, quod anima \blGetVerseX{9}{my note 1 at the margin}rationalis 
non est virtus \blGetVerseX{1}{my note 2 at the margin}in corpore, ita quod secundum esse vel
operationem vel utrumque ad \blGetVerseX{7}{my note 3 at the margin}corporis harmoniam dependeat.

\lettrine{125 }
autem ex multis \blGetVerse{5}et per alia media concludit idem. \blGetVerse{3}Hic enim primo ostendit, quod anima \blGetVerseX{9}{my note 1 at the margin}rationalis 
non est virtus \blGetVerseX{1}{my note 2 at the margin}in corpore, ita quod secundum esse vel
operationem vel utrumque ad \blGetVerseX{7}{my note 3 at the margin}corporis harmoniam dependeat.

\blSubTitle{Subtitulo numero Uno}

\lettrine{\blIniChar{FU}}{\blIniWord{turo}}
autem ex multis \blGetVerse{5}et per alia media concludit idem. \blGetVerse{3}Hic enim primo ostendit, quod anima \blGetVerseX{9}{my note 1 at the margin}rationalis 
non est virtus \blGetVerseX{1}{my note 2 at the margin}in corpore, ita quod secundum esse vel
operationem vel utrumque ad \blGetVerseX{7}{my note 3 at the margin}corporis harmoniam dependeat.


\blEndBook

\input{books/book3.tex}
\input{books/book4.tex}
\input{books/book5.tex}
\input{books/book6.tex}
\input{books/book7.tex}
\input{books/book8.tex}
\input{books/book9.tex}

\input{books/book10.tex}
\input{books/book11.tex}
\input{books/book12.tex}
\input{books/book13.tex}
\input{books/book14.tex}
\input{books/book15.tex}
\input{books/book16.tex}
\input{books/book17.tex}
\input{books/book18.tex}
\input{books/book19.tex}

\input{books/book20.tex}
\input{books/book21.tex}
\input{books/book22.tex}
\input{books/book23.tex}
\input{books/book24.tex}
\input{books/book25.tex}
\input{books/book26.tex}
\input{books/book27.tex}
\input{books/book28.tex}
\input{books/book29.tex}

\input{books/book30.tex}
\input{books/book31.tex}
\input{books/book32.tex}
\input{books/book33.tex}
\input{books/book34.tex}
\input{books/book35.tex}
\input{books/book36.tex}
\input{books/book37.tex}
\input{books/book38.tex}
\input{books/book39.tex}

\input{books/book40.tex}
\input{books/book41.tex}
\input{books/book42.tex}
\input{books/book43.tex}
\input{books/book44.tex}
\input{books/book45.tex}
\input{books/book46.tex}
\input{books/book47.tex}
\input{books/book48.tex}
\input{books/book49.tex}

\input{books/book50.tex}
\input{books/book51.tex}
\input{books/book52.tex}
\input{books/book53.tex}
\input{books/book54.tex}
\input{books/book55.tex}
\input{books/book56.tex}
\input{books/book57.tex}
\input{books/book58.tex}
\input{books/book59.tex}

\input{books/book60.tex}
\input{books/book61.tex}
\input{books/book62.tex}
\input{books/book63.tex}
\input{books/book64.tex}
\input{books/book65.tex}
\input{books/book66.tex}

%\end{multicols}
\end{document}

