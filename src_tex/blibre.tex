
\documentclass[twoside,12pt]{book}
\usepackage{polyglossia}
\usepackage{fontspec}
\usepackage{xunicode}
\usepackage{fancyhdr}
\usepackage{lettrine,microtype,xcolor}
\usepackage{geometry}
%\usepackage[showframe]{geometry}
%\usepackage[para,symbol*,bottom,flushmargin]{footmisc}
%\usepackage[para,symbol*,bottom]{footmisc}
%\usepackage[para,symbol*]{footmisc}
\usepackage[para,symbol*,flushmargin]{footmisc}
\usepackage{fourier-orns}
\usepackage{graphicx}
\usepackage{booktabs}
\usepackage{color}
\usepackage{multicol}
\usepackage[object=vectorian]{pgfornament}

\renewcommand{\familydefault}{pplj} 
\setdefaultlanguage{english}
\setotherlanguage{hebrew}

% \raggedbottom

% Footnote settings
%----------------------------------------------------

% The prefix 'bl' is used for all defined vars, commands, etc. meaning 'biblia-libre'

% Using the same symbol for all footnotes because:
%	1. many footnotes (like in a bible) always crash footnote counter.
%	2. footnote counter is reset every footnote.
%	3. footnotes must not use numbers to avoid confusing them with verse numbers.

\DefineFNsymbols*{blSymbols}{*****************************************************************
**********************************************************************************************}

\setfnsymbol{blSymbols}

% \interfootnotelinepenalty=0
% \addtolength{\skip\footins}{2pc plus 5pt}
% \setlength{\skip\footins}{5pt}

% Page layout.
%-----------------------------------------------------

\geometry{
	letterpaper,
	top = 20mm,
	bottom = 20mm,
	inner = 25mm,
	outer = 10mm,
	footskip = 8mm,
	headsep = 3mm,
	headheight = 20mm,
}

%\pagenumbering{gobble}

% Fonts to use. 
%-----------------------------------------------------

% in fonts folder
\newfontface\blSetDSHermannGotisch{DSHermannGotisch}
\newfontface\blSetDSRomantiques{DSRomantiques}
\newfontface\blSetWieynkFrakturVignetten{WieynkFrakturVignetten}
\newfontface\blSetRothenburgDecorative{RothenburgDecorative}

% in texlive full install
\input Acorn.fd
\newfontface\blSetAcorn{Acorn}

% in system (linux)
\newfontface\blSetDavidCLM{David CLM}

\newcommand{\blHebrewOne}[2]{\blSetDavidCLM\fontsize{#1}{40}\selectfont \RL{#2}}

% Some colors to use.
%-----------------------------------------------------
\definecolor{blMarron}{RGB}{60,30,10}
\definecolor{blDarkblue}{RGB}{0,0,80}
\definecolor{blLightblue}{RGB}{80,80,80}
\definecolor{blDarkgreen}{RGB}{0,80,0}
\definecolor{blDarkgray}{RGB}{0,80,0}
\definecolor{blDarkred}{RGB}{80,0,0}

% Header and Footer variables.
%-----------------------------------------------------

\newcommand{\blBookNameMeaning}{}
\newcommand{\blBookNumber}{MyBookNumber}
\newcommand{\blHebrewBookName}{MyHebrewBookName}
\newcommand{\blBookName}{MyBookName}
\newcommand{\blBookNameTranslit}{MyBookNameTranslit}

\newcommand{\blChapterNumber}{MyChapterNum}

% Variable setters y formated getters
%-----------------------------------------------------

\newcommand{\blSetBookNameMeaning}[1]{\renewcommand{\blBookNameMeaning}{#1}}
\newcommand{\blSetBookNumber}[1]{\renewcommand{\blBookNumber}{#1}}
\newcommand{\blSetHebrewBookName}[1]{\renewcommand{\blHebrewBookName}{#1}}
\newcommand{\blSetBookName}[1]{\renewcommand{\blBookName}{#1}}
\newcommand{\blSetBookNameTranslit}[1]{\renewcommand{\blBookNameTranslit}{#1}}

\newcommand{\blSetChapterNumber}[1]{\renewcommand{\blChapterNumber}{#1}}

\newcommand{\blGetVerse}[1]{%
\setcounter{footnote}{0}%
\textsuperscript{#1}%
}

\newcommand{\blGetVerseX}[2]{%
\footnote{#2}\blGetVerse{#1}%
}

\newcommand{\blIniChar}[1]{%
\blSetAcorn\selectfont{#1}%
}

\newcommand{\blIniWord}[1]{%
\blSetRothenburgDecorative\selectfont{#1}%
}

\newcommand{\blIniNum}[1]{%
\blSetDSRomantiques\selectfont{#1}%
}


% Header and Footer look.
%-----------------------------------------------------

\newcommand{\blOrnament}{ \vspace{3ex}\noindent \textcolor{blDarkgray}{\hrulefill~ \raisebox{-2.5pt}[10pt][10pt]{\leafright \decofourleft \decothreeleft  \aldineright \decotwo \floweroneleft \decoone   \floweroneright \decotwo \aldineleft\decothreeright \decofourright \leafleft} ~  \hrulefill \\ \vspace{3ex}}}

\newcommand{\blOddOrnament}{\noindent \textcolor{blDarkgray}{ \raisebox{-1.9pt}[10pt][10pt]{\leafright} \hrulefill \raisebox{-1.9pt}[10pt][10pt]{\leafright \decofourleft \decothreeleft  \aldineright \decotwo \floweroneleft \decoone}}}

\newcommand{\blEvenOrnament}{\noindent \textcolor{blDarkgray}{\raisebox{-1.9pt}[10pt][10pt]{\decoone \floweroneright \decotwo \aldineleft \decothreeright \decofourright \leafleft} \hrulefill \raisebox{-1.9pt}[10pt][10pt]{\leafleft}}}

\newcommand{\blTopOrnament}{\noindent \textcolor{blDarkgray}{\raisebox{-2.1pt}[10pt][10pt]{\leafright} \hrulefill \raisebox{-2.1pt}[10pt][10pt]{~~~\decofourleft \decotwo \decofourright~~~} \hrulefill \raisebox{-2.1pt}[10pt][10pt]{ \leafleft}} \vspace{0mm}}

\newcommand\blGetOrnament[3][black]{
	\resizebox{#2}{!}{\pgfornament[color = #1,ydelta=-40pt]{#3}}
}

\newcommand\blBookOrnament{{\blSetWieynkFrakturVignetten\fontsize{50}{40}\selectfont P}}

\newcommand\blGetBookTitle{
	{\textcolor{blDarkgreen}{\center{
		% \blGetOrnament[blDarkgreen]{0.7\linewidth}{85} \\
		\blOrnament
		 \hfill \hfill {\blHebrewOne{50}{\blHebrewBookName}} \hfill \blBookOrnament \\
		% \blGetOrnament[blDarkgreen]{0.2\linewidth}{71} \\
		\blBookNameMeaning \\ \blBookNameTranslit \\
		(\blBookName) \\
	}}}
	\vspace{1cm}
}


\makeatletter
\def\headrule{\blTopOrnament}
\makeatother

\pagestyle{fancy}

\fancyhf{}

\newcommand{\blTopInner}{\blBookName \space \blChapterNumber}
\newcommand{\blTopCenter}{\thepage}
\newcommand{\blTopOuter}{\blBookNameTranslit \space \blChapterNumber}

\newcommand{\blBottomOuter}{\blChapterNumber \space \blHebrewOne{12}{\blHebrewBookName}}
\newcommand{\blBottomOdd}{
	\blEvenOrnament \\ \large \hfill \sffamily\bf \textcolor{blDarkgray}{\leafNE ~~~ \blBottomOuter}
}
\newcommand{\blBottomEven}{
	\blOddOrnament \\ \large \sffamily\bf \textcolor{blDarkgray}{\blBottomOuter ~~~ \reflectbox{\leafNE}} \hfill
}

\fancyhead[LO,RE]{\blTopInner}
\fancyhead[CO,CE]{\blTopCenter}
\fancyhead[RO,LE]{\blTopOuter}

\fancyfoot[LO]{\blBottomOdd}
\fancyfoot[RE]{\blBottomEven}

\fancypagestyle{plain}{
	\fancyhead{}
	\renewcommand{\headrulewidth}{0pt}
	\renewcommand{\headrule}{}
}

\newcommand{\blStartBook}{%
	\thispagestyle{plain}%
	\blGetBookTitle%
}

\newcommand{\blEndBook}{%
	\center{\textcolor{blDarkgray}{\reflectbox{\blBookOrnament}\blBookOrnament}}%
	\cleardoublepage%
}

\newcommand{\blSubTitle}[1]{
	\vspace{2ex}
	\centerline{\textcolor{blDarkgray}{\textbf{\fontsize{16}{40}\selectfont #1}}}%
}


%====================================================================================================
%START OF DOCUMENT
%====================================================================================================
\begin{document}
%\begin{multicols}{2}

%\thispagestyle{empty}


\blSetBookNameMeaning{En el comienzo}
\blSetBookNumber{1}
\blSetHebrewBookName{בראשית}
\blSetBookName{Genesis}
\blSetBookNameTranslit{Bereshit}
\blStartBook

\blSetChapterNumber{1}

%{\blSetAcorn\selectfont ABCDEF GHIJKL MNÑOPQ RSTUVW XYZ 01234 56789}

\lettrine{\blIniChar{FU}}{\blIniWord{turo}}
autem ex multis \blGetVerse{5}et per alia media concludit idem. \blGetVerse{3}Hic enim primo ostendit, quod anima \blGetVerseX{9}{my note 1 at the margin}rationalis 
non est virtus \blGetVerseX{1}{my note 2 at the margin}in corpore, ita quod secundum esse vel
operationem vel utrumque ad \blGetVerseX{7}{my note 3 at the margin}corporis harmoniam dependeat.

\lettrine{\blIniNum{150}}{Future}
autem ex multis \blGetVerse{5}et per alia media concludit idem. \blGetVerse{3}Hic enim primo ostendit, quod anima \blGetVerseX{9}{my note 1 at the margin}rationalis 
non est virtus \blGetVerseX{1}{my note 2 at the margin}in corpore, ita quod secundum esse vel
operationem vel utrumque ad \blGetVerseX{7}{my note 3 at the margin}corporis harmoniam dependeat.

\lettrine{125 }
autem ex multis \blGetVerse{5}et per alia media concludit idem. \blGetVerse{3}Hic enim primo ostendit, quod anima \blGetVerseX{9}{my note 1 at the margin}rationalis 
non est virtus \blGetVerseX{1}{my note 2 at the margin}in corpore, ita quod secundum esse vel
operationem vel utrumque ad \blGetVerseX{7}{my note 3 at the margin}corporis harmoniam dependeat.

\blSubTitle{Subtitulo numero Uno}

\lettrine{\blIniChar{FU}}{\blIniWord{turo}}
autem ex multis \blGetVerse{5}et per alia media concludit idem. \blGetVerse{3}Hic enim primo ostendit, quod anima \blGetVerseX{9}{my note 1 at the margin}rationalis 
non est virtus \blGetVerseX{1}{my note 2 at the margin}in corpore, ita quod secundum esse vel
operationem vel utrumque ad \blGetVerseX{7}{my note 3 at the margin}corporis harmoniam dependeat.

\lettrine{\blIniNum{150}}{Future}
autem ex multis \blGetVerse{5}et per alia media concludit idem. \blGetVerse{3}Hic enim primo ostendit, quod anima \blGetVerseX{9}{my note 1 at the margin}rationalis 
non est virtus \blGetVerseX{1}{my note 2 at the margin}in corpore, ita quod secundum esse vel
operationem vel utrumque ad \blGetVerseX{7}{my note 3 at the margin}corporis harmoniam dependeat.

\lettrine{125 }
autem ex multis \blGetVerse{5}et per alia media concludit idem. \blGetVerse{3}Hic enim primo ostendit, quod anima \blGetVerseX{9}{my note 1 at the margin}rationalis 
non est virtus \blGetVerseX{1}{my note 2 at the margin}in corpore, ita quod secundum esse vel
operationem vel utrumque ad \blGetVerseX{7}{my note 3 at the margin}corporis harmoniam dependeat.

\lettrine{\blIniChar{FU}}{\blIniWord{turo}}
autem ex multis \blGetVerse{5}et per alia media concludit idem. \blGetVerse{3}Hic enim primo ostendit, quod anima \blGetVerseX{9}{my note 1 at the margin}rationalis 
non est virtus \blGetVerseX{1}{my note 2 at the margin}in corpore, ita quod secundum esse vel
operationem vel utrumque ad \blGetVerseX{7}{my note 3 at the margin}corporis harmoniam dependeat.

\lettrine{\blIniNum{150}}{Future}
autem ex multis \blGetVerse{5}et per alia media concludit idem. \blGetVerse{3}Hic enim primo ostendit, quod anima \blGetVerseX{9}{my note 1 at the margin}rationalis 
non est virtus \blGetVerseX{1}{my note 2 at the margin}in corpore, ita quod secundum esse vel
operationem vel utrumque ad \blGetVerseX{7}{my note 3 at the margin}corporis harmoniam dependeat.

\lettrine{125 }
autem ex multis \blGetVerse{5}et per alia media concludit idem. \blGetVerse{3}Hic enim primo ostendit, quod anima \blGetVerseX{9}{my note 1 at the margin}rationalis 
non est virtus \blGetVerseX{1}{my note 2 at the margin}in corpore, ita quod secundum esse vel
operationem vel utrumque ad \blGetVerseX{7}{my note 3 at the margin}corporis harmoniam dependeat.

\lettrine{\blIniChar{FU}}{\blIniWord{turo}}
autem ex multis \blGetVerse{5}et per alia media concludit idem. \blGetVerse{3}Hic enim primo ostendit, quod anima \blGetVerseX{9}{my note 1 at the margin}rationalis 
non est virtus \blGetVerseX{1}{my note 2 at the margin}in corpore, ita quod secundum esse vel
operationem vel utrumque ad \blGetVerseX{7}{my note 3 at the margin}corporis harmoniam dependeat.

\lettrine{\blIniNum{150}}{Future}
autem ex multis \blGetVerse{5}et per alia media concludit idem. \blGetVerse{3}Hic enim primo ostendit, quod anima \blGetVerseX{9}{my note 1 at the margin}rationalis 
non est virtus \blGetVerseX{1}{my note 2 at the margin}in corpore, ita quod secundum esse vel
operationem vel utrumque ad \blGetVerseX{7}{my note 3 at the margin}corporis harmoniam dependeat.

\lettrine{125 }
autem ex multis \blGetVerse{5}et per alia media concludit idem. \blGetVerse{3}Hic enim primo ostendit, quod anima \blGetVerseX{9}{my note 1 at the margin}rationalis 
non est virtus \blGetVerseX{1}{my note 2 at the margin}in corpore, ita quod secundum esse vel
operationem vel utrumque ad \blGetVerseX{7}{my note 3 at the margin}corporis harmoniam dependeat.

\lettrine{\blIniChar{FU}}{\blIniWord{turo}}
autem ex multis \blGetVerse{5}et per alia media concludit idem. \blGetVerse{3}Hic enim primo ostendit, quod anima \blGetVerseX{9}{my note 1 at the margin}rationalis 
non est virtus \blGetVerseX{1}{my note 2 at the margin}in corpore, ita quod secundum esse vel
operationem vel utrumque ad \blGetVerseX{7}{my note 3 at the margin}corporis harmoniam dependeat.

\lettrine{\blIniNum{150}}{Future}
autem ex multis \blGetVerse{5}et per alia media concludit idem. \blGetVerse{3}Hic enim primo ostendit, quod anima \blGetVerseX{9}{my note 1 at the margin}rationalis 
non est virtus \blGetVerseX{1}{my note 2 at the margin}in corpore, ita quod secundum esse vel
operationem vel utrumque ad \blGetVerseX{7}{my note 3 at the margin}corporis harmoniam dependeat.

\lettrine{125 }
autem ex multis \blGetVerse{5}et per alia media concludit idem. \blGetVerse{3}Hic enim primo ostendit, quod anima \blGetVerseX{9}{my note 1 at the margin}rationalis 
non est virtus \blGetVerseX{1}{my note 2 at the margin}in corpore, ita quod secundum esse vel
operationem vel utrumque ad \blGetVerseX{7}{my note 3 at the margin}corporis harmoniam dependeat.

\lettrine{\blIniChar{FU}}{\blIniWord{turo}}
autem ex multis \blGetVerse{5}et per alia media concludit idem. \blGetVerse{3}Hic enim primo ostendit, quod anima \blGetVerseX{9}{my note 1 at the margin}rationalis 
non est virtus \blGetVerseX{1}{my note 2 at the margin}in corpore, ita quod secundum esse vel
operationem vel utrumque ad \blGetVerseX{7}{my note 3 at the margin}corporis harmoniam dependeat.

\lettrine{\blIniNum{150}}{Future}
autem ex multis \blGetVerse{5}et per alia media concludit idem. \blGetVerse{3}Hic enim primo ostendit, quod anima \blGetVerseX{9}{my note 1 at the margin}rationalis 
non est virtus \blGetVerseX{1}{my note 2 at the margin}in corpore, ita quod secundum esse vel
operationem vel utrumque ad \blGetVerseX{7}{my note 3 at the margin}corporis harmoniam dependeat.

\lettrine{125 }
autem ex multis \blGetVerse{5}et per alia media concludit idem. \blGetVerse{3}Hic enim primo ostendit, quod anima \blGetVerseX{9}{my note 1 at the margin}rationalis 
non est virtus \blGetVerseX{1}{my note 2 at the margin}in corpore, ita quod secundum esse vel
operationem vel utrumque ad \blGetVerseX{7}{my note 3 at the margin}corporis harmoniam dependeat.

\lettrine{\blIniChar{FU}}{\blIniWord{turo}}
autem ex multis \blGetVerse{5}et per alia media concludit idem. \blGetVerse{3}Hic enim primo ostendit, quod anima \blGetVerseX{9}{my note 1 at the margin}rationalis 
non est virtus \blGetVerseX{1}{my note 2 at the margin}in corpore, ita quod secundum esse vel
operationem vel utrumque ad \blGetVerseX{7}{my note 3 at the margin}corporis harmoniam dependeat.

\lettrine{\blIniNum{150}}{Future}
autem ex multis \blGetVerse{5}et per alia media concludit idem. \blGetVerse{3}Hic enim primo ostendit, quod anima \blGetVerseX{9}{my note 1 at the margin}rationalis 
non est virtus \blGetVerseX{1}{my note 2 at the margin}in corpore, ita quod secundum esse vel
operationem vel utrumque ad \blGetVerseX{7}{my note 3 at the margin}corporis harmoniam dependeat.

\lettrine{125 }
autem ex multis \blGetVerse{5}et per alia media concludit idem. \blGetVerse{3}Hic enim primo ostendit, quod anima \blGetVerseX{9}{my note 1 at the margin}rationalis 
non est virtus \blGetVerseX{1}{my note 2 at the margin}in corpore, ita quod secundum esse vel
operationem vel utrumque ad \blGetVerseX{7}{my note 3 at the margin}corporis harmoniam dependeat.

\lettrine{\blIniChar{FU}}{\blIniWord{turo}}
autem ex multis \blGetVerse{5}et per alia media concludit idem. \blGetVerse{3}Hic enim primo ostendit, quod anima \blGetVerseX{9}{my note 1 at the margin}rationalis 
non est virtus \blGetVerseX{1}{my note 2 at the margin}in corpore, ita quod secundum esse vel
operationem vel utrumque ad \blGetVerseX{7}{my note 3 at the margin}corporis harmoniam dependeat.

\lettrine{\blIniNum{150}}{Future}
autem ex multis \blGetVerse{5}et per alia media concludit idem. \blGetVerse{3}Hic enim primo ostendit, quod anima \blGetVerseX{9}{my note 1 at the margin}rationalis 
non est virtus \blGetVerseX{1}{my note 2 at the margin}in corpore, ita quod secundum esse vel
operationem vel utrumque ad \blGetVerseX{7}{my note 3 at the margin}corporis harmoniam dependeat.

\lettrine{125 }
autem ex multis \blGetVerse{5}et per alia media concludit idem. \blGetVerse{3}Hic enim primo ostendit, quod anima \blGetVerseX{9}{my note 1 at the margin}rationalis 
non est virtus \blGetVerseX{1}{my note 2 at the margin}in corpore, ita quod secundum esse vel
operationem vel utrumque ad \blGetVerseX{7}{my note 3 at the margin}corporis harmoniam dependeat.

\lettrine{\blIniChar{FU}}{\blIniWord{turo}}
autem ex multis \blGetVerse{5}et per alia media concludit idem. \blGetVerse{3}Hic enim primo ostendit, quod anima \blGetVerseX{9}{my note 1 at the margin}rationalis 
non est virtus \blGetVerseX{1}{my note 2 at the margin}in corpore, ita quod secundum esse vel
operationem vel utrumque ad \blGetVerseX{7}{my note 3 at the margin}corporis harmoniam dependeat.

\lettrine{\blIniNum{150}}{Future}
autem ex multis \blGetVerse{5}et per alia media concludit idem. \blGetVerse{3}Hic enim primo ostendit, quod anima \blGetVerseX{9}{my note 1 at the margin}rationalis 
non est virtus \blGetVerseX{1}{my note 2 at the margin}in corpore, ita quod secundum esse vel
operationem vel utrumque ad \blGetVerseX{7}{my note 3 at the margin}corporis harmoniam dependeat.

\lettrine{125 }
autem ex multis \blGetVerse{5}et per alia media concludit idem. \blGetVerse{3}Hic enim primo ostendit, quod anima \blGetVerseX{9}{my note 1 at the margin}rationalis 
non est virtus \blGetVerseX{1}{my note 2 at the margin}in corpore, ita quod secundum esse vel
operationem vel utrumque ad \blGetVerseX{7}{my note 3 at the margin}corporis harmoniam dependeat.

\lettrine{\blIniChar{FU}}{\blIniWord{turo}}
autem ex multis \blGetVerse{5}et per alia media concludit idem. \blGetVerse{3}Hic enim primo ostendit, quod anima \blGetVerseX{9}{my note 1 at the margin}rationalis 
non est virtus \blGetVerseX{1}{my note 2 at the margin}in corpore, ita quod secundum esse vel
operationem vel utrumque ad \blGetVerseX{7}{my note 3 at the margin}corporis harmoniam dependeat.

\lettrine{\blIniNum{150}}{Future}
autem ex multis \blGetVerse{5}et per alia media concludit idem. \blGetVerse{3}Hic enim primo ostendit, quod anima \blGetVerseX{9}{my note 1 at the margin}rationalis 
non est virtus \blGetVerseX{1}{my note 2 at the margin}in corpore, ita quod secundum esse vel
operationem vel utrumque ad \blGetVerseX{7}{my note 3 at the margin}corporis harmoniam dependeat.

\lettrine{125 }
autem ex multis \blGetVerse{5}et per alia media concludit idem. \blGetVerse{3}Hic enim primo ostendit, quod anima \blGetVerseX{9}{my note 1 at the margin}rationalis 
non est virtus \blGetVerseX{1}{my note 2 at the margin}in corpore, ita quod secundum esse vel
operationem vel utrumque ad \blGetVerseX{7}{my note 3 at the margin}corporis harmoniam dependeat.

\lettrine{\blIniChar{FU}}{\blIniWord{turo}}
autem ex multis \blGetVerse{5}et per alia media concludit idem. \blGetVerse{3}Hic enim primo ostendit, quod anima \blGetVerseX{9}{my note 1 at the margin}rationalis 
non est virtus \blGetVerseX{1}{my note 2 at the margin}in corpore, ita quod secundum esse vel
operationem vel utrumque ad \blGetVerseX{7}{my note 3 at the margin}corporis harmoniam dependeat.

\lettrine{\blIniNum{150}}{Future}
autem ex multis \blGetVerse{5}et per alia media concludit idem. \blGetVerse{3}Hic enim primo ostendit, quod anima \blGetVerseX{9}{my note 1 at the margin}rationalis 
non est virtus \blGetVerseX{1}{my note 2 at the margin}in corpore, ita quod secundum esse vel
operationem vel utrumque ad \blGetVerseX{7}{my note 3 at the margin}corporis harmoniam dependeat.

\lettrine{125 }
autem ex multis \blGetVerse{5}et per alia media concludit idem. \blGetVerse{3}Hic enim primo ostendit, quod anima \blGetVerseX{9}{my note 1 at the margin}rationalis 
non est virtus \blGetVerseX{1}{my note 2 at the margin}in corpore, ita quod secundum esse vel
operationem vel utrumque ad \blGetVerseX{7}{my note 3 at the margin}corporis harmoniam dependeat.

\lettrine{\blIniChar{FU}}{\blIniWord{turo}}
autem ex multis \blGetVerse{5}et per alia media concludit idem. \blGetVerse{3}Hic enim primo ostendit, quod anima \blGetVerseX{9}{my note 1 at the margin}rationalis 
non est virtus \blGetVerseX{1}{my note 2 at the margin}in corpore, ita quod secundum esse vel
operationem vel utrumque ad \blGetVerseX{7}{my note 3 at the margin}corporis harmoniam dependeat.

\lettrine{\blIniNum{150}}{Future}
autem ex multis \blGetVerse{5}et per alia media concludit idem. \blGetVerse{3}Hic enim primo ostendit, quod anima \blGetVerseX{9}{my note 1 at the margin}rationalis 
non est virtus \blGetVerseX{1}{my note 2 at the margin}in corpore, ita quod secundum esse vel
operationem vel utrumque ad \blGetVerseX{7}{my note 3 at the margin}corporis harmoniam dependeat.

\lettrine{125 }
autem ex multis \blGetVerse{5}et per alia media concludit idem. \blGetVerse{3}Hic enim primo ostendit, quod anima \blGetVerseX{9}{my note 1 at the margin}rationalis 
non est virtus \blGetVerseX{1}{my note 2 at the margin}in corpore, ita quod secundum esse vel
operationem vel utrumque ad \blGetVerseX{7}{my note 3 at the margin}corporis harmoniam dependeat.

\lettrine{\blIniChar{FU}}{\blIniWord{turo}}
autem ex multis \blGetVerse{5}et per alia media concludit idem. \blGetVerse{3}Hic enim primo ostendit, quod anima \blGetVerseX{9}{my note 1 at the margin}rationalis 
non est virtus \blGetVerseX{1}{my note 2 at the margin}in corpore, ita quod secundum esse vel
operationem vel utrumque ad \blGetVerseX{7}{my note 3 at the margin}corporis harmoniam dependeat.

\lettrine{\blIniNum{150}}{Future}
autem ex multis \blGetVerse{5}et per alia media concludit idem. \blGetVerse{3}Hic enim primo ostendit, quod anima \blGetVerseX{9}{my note 1 at the margin}rationalis 
non est virtus \blGetVerseX{1}{my note 2 at the margin}in corpore, ita quod secundum esse vel
operationem vel utrumque ad \blGetVerseX{7}{my note 3 at the margin}corporis harmoniam dependeat.

\lettrine{125 }
autem ex multis \blGetVerse{5}et per alia media concludit idem. \blGetVerse{3}Hic enim primo ostendit, quod anima \blGetVerseX{9}{my note 1 at the margin}rationalis 
non est virtus \blGetVerseX{1}{my note 2 at the margin}in corpore, ita quod secundum esse vel
operationem vel utrumque ad \blGetVerseX{7}{my note 3 at the margin}corporis harmoniam dependeat.

\lettrine{\blIniChar{FU}}{\blIniWord{turo}}
autem ex multis \blGetVerse{5}et per alia media concludit idem. \blGetVerse{3}Hic enim primo ostendit, quod anima \blGetVerseX{9}{my note 1 at the margin}rationalis 
non est virtus \blGetVerseX{1}{my note 2 at the margin}in corpore, ita quod secundum esse vel
operationem vel utrumque ad \blGetVerseX{7}{my note 3 at the margin}corporis harmoniam dependeat.

\lettrine{\blIniNum{150}}{Future}
autem ex multis \blGetVerse{5}et per alia media concludit idem. \blGetVerse{3}Hic enim primo ostendit, quod anima \blGetVerseX{9}{my note 1 at the margin}rationalis 
non est virtus \blGetVerseX{1}{my note 2 at the margin}in corpore, ita quod secundum esse vel
operationem vel utrumque ad \blGetVerseX{7}{my note 3 at the margin}corporis harmoniam dependeat.

\lettrine{125 }
autem ex multis \blGetVerse{5}et per alia media concludit idem. \blGetVerse{3}Hic enim primo ostendit, quod anima \blGetVerseX{9}{my note 1 at the margin}rationalis 
non est virtus \blGetVerseX{1}{my note 2 at the margin}in corpore, ita quod secundum esse vel
operationem vel utrumque ad \blGetVerseX{7}{my note 3 at the margin}corporis harmoniam dependeat.

\lettrine{\blIniChar{FU}}{\blIniWord{turo}}
autem ex multis \blGetVerse{5}et per alia media concludit idem. \blGetVerse{3}Hic enim primo ostendit, quod anima \blGetVerseX{9}{my note 1 at the margin}rationalis 
non est virtus \blGetVerseX{1}{my note 2 at the margin}in corpore, ita quod secundum esse vel
operationem vel utrumque ad \blGetVerseX{7}{my note 3 at the margin}corporis harmoniam dependeat.

\lettrine{\blIniNum{150}}{Future}
autem ex multis \blGetVerse{5}et per alia media concludit idem. \blGetVerse{3}Hic enim primo ostendit, quod anima \blGetVerseX{9}{my note 1 at the margin}rationalis 
non est virtus \blGetVerseX{1}{my note 2 at the margin}in corpore, ita quod secundum esse vel
operationem vel utrumque ad \blGetVerseX{7}{my note 3 at the margin}corporis harmoniam dependeat.

\lettrine{125 }
autem ex multis \blGetVerse{5}et per alia media concludit idem. \blGetVerse{3}Hic enim primo ostendit, quod anima \blGetVerseX{9}{my note 1 at the margin}rationalis 
non est virtus \blGetVerseX{1}{my note 2 at the margin}in corpore, ita quod secundum esse vel
operationem vel utrumque ad \blGetVerseX{7}{my note 3 at the margin}corporis harmoniam dependeat.

\lettrine{\blIniChar{FU}}{\blIniWord{turo}}
autem ex multis \blGetVerse{5}et per alia media concludit idem. \blGetVerse{3}Hic enim primo ostendit, quod anima \blGetVerseX{9}{my note 1 at the margin}rationalis 
non est virtus \blGetVerseX{1}{my note 2 at the margin}in corpore, ita quod secundum esse vel
operationem vel utrumque ad \blGetVerseX{7}{my note 3 at the margin}corporis harmoniam dependeat.

\lettrine{\blIniNum{150}}{Future}
autem ex multis \blGetVerse{5}et per alia media concludit idem. \blGetVerse{3}Hic enim primo ostendit, quod anima \blGetVerseX{9}{my note 1 at the margin}rationalis 
non est virtus \blGetVerseX{1}{my note 2 at the margin}in corpore, ita quod secundum esse vel
operationem vel utrumque ad \blGetVerseX{7}{my note 3 at the margin}corporis harmoniam dependeat.

\lettrine{125 }
autem ex multis \blGetVerse{5}et per alia media concludit idem. \blGetVerse{3}Hic enim primo ostendit, quod anima \blGetVerseX{9}{my note 1 at the margin}rationalis 
non est virtus \blGetVerseX{1}{my note 2 at the margin}in corpore, ita quod secundum esse vel
operationem vel utrumque ad \blGetVerseX{7}{my note 3 at the margin}corporis harmoniam dependeat.


\blEndBook



\blSetBookNameMeaning{En el comienzo}
\blSetBookNumber{1}
\blSetHebrewBookName{בראשית}
\blSetBookName{Genesis}
\blSetBookNameTranslit{Bereshit}
\blStartBook

\blSetChapterNumber{1}

%{\blSetAcorn\selectfont ABCDEF GHIJKL MNÑOPQ RSTUVW XYZ 01234 56789}

\lettrine{\blIniChar{FU}}{\blIniWord{turo}}
autem ex multis \blGetVerse{5}et per alia media concludit idem. \blGetVerse{3}Hic enim primo ostendit, quod anima \blGetVerseX{9}{my note 1 at the margin}rationalis 
non est virtus \blGetVerseX{1}{my note 2 at the margin}in corpore, ita quod secundum esse vel
operationem vel utrumque ad \blGetVerseX{7}{my note 3 at the margin}corporis harmoniam dependeat.

\lettrine{\blIniNum{150}}{Future}
autem ex multis \blGetVerse{5}et per alia media concludit idem. \blGetVerse{3}Hic enim primo ostendit, quod anima \blGetVerseX{9}{my note 1 at the margin}rationalis 
non est virtus \blGetVerseX{1}{my note 2 at the margin}in corpore, ita quod secundum esse vel
operationem vel utrumque ad \blGetVerseX{7}{my note 3 at the margin}corporis harmoniam dependeat.

\lettrine{125 }
autem ex multis \blGetVerse{5}et per alia media concludit idem. \blGetVerse{3}Hic enim primo ostendit, quod anima \blGetVerseX{9}{my note 1 at the margin}rationalis 
non est virtus \blGetVerseX{1}{my note 2 at the margin}in corpore, ita quod secundum esse vel
operationem vel utrumque ad \blGetVerseX{7}{my note 3 at the margin}corporis harmoniam dependeat.

\blSubTitle{Subtitulo numero Uno}

\lettrine{\blIniChar{FU}}{\blIniWord{turo}}
autem ex multis \blGetVerse{5}et per alia media concludit idem. \blGetVerse{3}Hic enim primo ostendit, quod anima \blGetVerseX{9}{my note 1 at the margin}rationalis 
non est virtus \blGetVerseX{1}{my note 2 at the margin}in corpore, ita quod secundum esse vel
operationem vel utrumque ad \blGetVerseX{7}{my note 3 at the margin}corporis harmoniam dependeat.

\lettrine{\blIniNum{150}}{Future}
autem ex multis \blGetVerse{5}et per alia media concludit idem. \blGetVerse{3}Hic enim primo ostendit, quod anima \blGetVerseX{9}{my note 1 at the margin}rationalis 
non est virtus \blGetVerseX{1}{my note 2 at the margin}in corpore, ita quod secundum esse vel
operationem vel utrumque ad \blGetVerseX{7}{my note 3 at the margin}corporis harmoniam dependeat.

\lettrine{125 }
autem ex multis \blGetVerse{5}et per alia media concludit idem. \blGetVerse{3}Hic enim primo ostendit, quod anima \blGetVerseX{9}{my note 1 at the margin}rationalis 
non est virtus \blGetVerseX{1}{my note 2 at the margin}in corpore, ita quod secundum esse vel
operationem vel utrumque ad \blGetVerseX{7}{my note 3 at the margin}corporis harmoniam dependeat.

\lettrine{\blIniChar{FU}}{\blIniWord{turo}}
autem ex multis \blGetVerse{5}et per alia media concludit idem. \blGetVerse{3}Hic enim primo ostendit, quod anima \blGetVerseX{9}{my note 1 at the margin}rationalis 
non est virtus \blGetVerseX{1}{my note 2 at the margin}in corpore, ita quod secundum esse vel
operationem vel utrumque ad \blGetVerseX{7}{my note 3 at the margin}corporis harmoniam dependeat.

\lettrine{\blIniNum{150}}{Future}
autem ex multis \blGetVerse{5}et per alia media concludit idem. \blGetVerse{3}Hic enim primo ostendit, quod anima \blGetVerseX{9}{my note 1 at the margin}rationalis 
non est virtus \blGetVerseX{1}{my note 2 at the margin}in corpore, ita quod secundum esse vel
operationem vel utrumque ad \blGetVerseX{7}{my note 3 at the margin}corporis harmoniam dependeat.

\lettrine{125 }
autem ex multis \blGetVerse{5}et per alia media concludit idem. \blGetVerse{3}Hic enim primo ostendit, quod anima \blGetVerseX{9}{my note 1 at the margin}rationalis 
non est virtus \blGetVerseX{1}{my note 2 at the margin}in corpore, ita quod secundum esse vel
operationem vel utrumque ad \blGetVerseX{7}{my note 3 at the margin}corporis harmoniam dependeat.

\lettrine{\blIniChar{FU}}{\blIniWord{turo}}
autem ex multis \blGetVerse{5}et per alia media concludit idem. \blGetVerse{3}Hic enim primo ostendit, quod anima \blGetVerseX{9}{my note 1 at the margin}rationalis 
non est virtus \blGetVerseX{1}{my note 2 at the margin}in corpore, ita quod secundum esse vel
operationem vel utrumque ad \blGetVerseX{7}{my note 3 at the margin}corporis harmoniam dependeat.

\lettrine{\blIniNum{150}}{Future}
autem ex multis \blGetVerse{5}et per alia media concludit idem. \blGetVerse{3}Hic enim primo ostendit, quod anima \blGetVerseX{9}{my note 1 at the margin}rationalis 
non est virtus \blGetVerseX{1}{my note 2 at the margin}in corpore, ita quod secundum esse vel
operationem vel utrumque ad \blGetVerseX{7}{my note 3 at the margin}corporis harmoniam dependeat.

\lettrine{125 }
autem ex multis \blGetVerse{5}et per alia media concludit idem. \blGetVerse{3}Hic enim primo ostendit, quod anima \blGetVerseX{9}{my note 1 at the margin}rationalis 
non est virtus \blGetVerseX{1}{my note 2 at the margin}in corpore, ita quod secundum esse vel
operationem vel utrumque ad \blGetVerseX{7}{my note 3 at the margin}corporis harmoniam dependeat.

\lettrine{\blIniChar{FU}}{\blIniWord{turo}}
autem ex multis \blGetVerse{5}et per alia media concludit idem. \blGetVerse{3}Hic enim primo ostendit, quod anima \blGetVerseX{9}{my note 1 at the margin}rationalis 
non est virtus \blGetVerseX{1}{my note 2 at the margin}in corpore, ita quod secundum esse vel
operationem vel utrumque ad \blGetVerseX{7}{my note 3 at the margin}corporis harmoniam dependeat.

\lettrine{\blIniNum{150}}{Future}
autem ex multis \blGetVerse{5}et per alia media concludit idem. \blGetVerse{3}Hic enim primo ostendit, quod anima \blGetVerseX{9}{my note 1 at the margin}rationalis 
non est virtus \blGetVerseX{1}{my note 2 at the margin}in corpore, ita quod secundum esse vel
operationem vel utrumque ad \blGetVerseX{7}{my note 3 at the margin}corporis harmoniam dependeat.

\lettrine{125 }
autem ex multis \blGetVerse{5}et per alia media concludit idem. \blGetVerse{3}Hic enim primo ostendit, quod anima \blGetVerseX{9}{my note 1 at the margin}rationalis 
non est virtus \blGetVerseX{1}{my note 2 at the margin}in corpore, ita quod secundum esse vel
operationem vel utrumque ad \blGetVerseX{7}{my note 3 at the margin}corporis harmoniam dependeat.

\lettrine{\blIniChar{FU}}{\blIniWord{turo}}
autem ex multis \blGetVerse{5}et per alia media concludit idem. \blGetVerse{3}Hic enim primo ostendit, quod anima \blGetVerseX{9}{my note 1 at the margin}rationalis 
non est virtus \blGetVerseX{1}{my note 2 at the margin}in corpore, ita quod secundum esse vel
operationem vel utrumque ad \blGetVerseX{7}{my note 3 at the margin}corporis harmoniam dependeat.

\lettrine{\blIniNum{150}}{Future}
autem ex multis \blGetVerse{5}et per alia media concludit idem. \blGetVerse{3}Hic enim primo ostendit, quod anima \blGetVerseX{9}{my note 1 at the margin}rationalis 
non est virtus \blGetVerseX{1}{my note 2 at the margin}in corpore, ita quod secundum esse vel
operationem vel utrumque ad \blGetVerseX{7}{my note 3 at the margin}corporis harmoniam dependeat.

\lettrine{125 }
autem ex multis \blGetVerse{5}et per alia media concludit idem. \blGetVerse{3}Hic enim primo ostendit, quod anima \blGetVerseX{9}{my note 1 at the margin}rationalis 
non est virtus \blGetVerseX{1}{my note 2 at the margin}in corpore, ita quod secundum esse vel
operationem vel utrumque ad \blGetVerseX{7}{my note 3 at the margin}corporis harmoniam dependeat.

\lettrine{\blIniChar{FU}}{\blIniWord{turo}}
autem ex multis \blGetVerse{5}et per alia media concludit idem. \blGetVerse{3}Hic enim primo ostendit, quod anima \blGetVerseX{9}{my note 1 at the margin}rationalis 
non est virtus \blGetVerseX{1}{my note 2 at the margin}in corpore, ita quod secundum esse vel
operationem vel utrumque ad \blGetVerseX{7}{my note 3 at the margin}corporis harmoniam dependeat.

\lettrine{\blIniNum{150}}{Future}
autem ex multis \blGetVerse{5}et per alia media concludit idem. \blGetVerse{3}Hic enim primo ostendit, quod anima \blGetVerseX{9}{my note 1 at the margin}rationalis 
non est virtus \blGetVerseX{1}{my note 2 at the margin}in corpore, ita quod secundum esse vel
operationem vel utrumque ad \blGetVerseX{7}{my note 3 at the margin}corporis harmoniam dependeat.

\lettrine{125 }
autem ex multis \blGetVerse{5}et per alia media concludit idem. \blGetVerse{3}Hic enim primo ostendit, quod anima \blGetVerseX{9}{my note 1 at the margin}rationalis 
non est virtus \blGetVerseX{1}{my note 2 at the margin}in corpore, ita quod secundum esse vel
operationem vel utrumque ad \blGetVerseX{7}{my note 3 at the margin}corporis harmoniam dependeat.

\lettrine{\blIniChar{FU}}{\blIniWord{turo}}
autem ex multis \blGetVerse{5}et per alia media concludit idem. \blGetVerse{3}Hic enim primo ostendit, quod anima \blGetVerseX{9}{my note 1 at the margin}rationalis 
non est virtus \blGetVerseX{1}{my note 2 at the margin}in corpore, ita quod secundum esse vel
operationem vel utrumque ad \blGetVerseX{7}{my note 3 at the margin}corporis harmoniam dependeat.

\lettrine{\blIniNum{150}}{Future}
autem ex multis \blGetVerse{5}et per alia media concludit idem. \blGetVerse{3}Hic enim primo ostendit, quod anima \blGetVerseX{9}{my note 1 at the margin}rationalis 
non est virtus \blGetVerseX{1}{my note 2 at the margin}in corpore, ita quod secundum esse vel
operationem vel utrumque ad \blGetVerseX{7}{my note 3 at the margin}corporis harmoniam dependeat.

\lettrine{125 }
autem ex multis \blGetVerse{5}et per alia media concludit idem. \blGetVerse{3}Hic enim primo ostendit, quod anima \blGetVerseX{9}{my note 1 at the margin}rationalis 
non est virtus \blGetVerseX{1}{my note 2 at the margin}in corpore, ita quod secundum esse vel
operationem vel utrumque ad \blGetVerseX{7}{my note 3 at the margin}corporis harmoniam dependeat.

\lettrine{\blIniChar{FU}}{\blIniWord{turo}}
autem ex multis \blGetVerse{5}et per alia media concludit idem. \blGetVerse{3}Hic enim primo ostendit, quod anima \blGetVerseX{9}{my note 1 at the margin}rationalis 
non est virtus \blGetVerseX{1}{my note 2 at the margin}in corpore, ita quod secundum esse vel
operationem vel utrumque ad \blGetVerseX{7}{my note 3 at the margin}corporis harmoniam dependeat.

\lettrine{\blIniNum{150}}{Future}
autem ex multis \blGetVerse{5}et per alia media concludit idem. \blGetVerse{3}Hic enim primo ostendit, quod anima \blGetVerseX{9}{my note 1 at the margin}rationalis 
non est virtus \blGetVerseX{1}{my note 2 at the margin}in corpore, ita quod secundum esse vel
operationem vel utrumque ad \blGetVerseX{7}{my note 3 at the margin}corporis harmoniam dependeat.

\lettrine{125 }
autem ex multis \blGetVerse{5}et per alia media concludit idem. \blGetVerse{3}Hic enim primo ostendit, quod anima \blGetVerseX{9}{my note 1 at the margin}rationalis 
non est virtus \blGetVerseX{1}{my note 2 at the margin}in corpore, ita quod secundum esse vel
operationem vel utrumque ad \blGetVerseX{7}{my note 3 at the margin}corporis harmoniam dependeat.

\lettrine{\blIniChar{FU}}{\blIniWord{turo}}
autem ex multis \blGetVerse{5}et per alia media concludit idem. \blGetVerse{3}Hic enim primo ostendit, quod anima \blGetVerseX{9}{my note 1 at the margin}rationalis 
non est virtus \blGetVerseX{1}{my note 2 at the margin}in corpore, ita quod secundum esse vel
operationem vel utrumque ad \blGetVerseX{7}{my note 3 at the margin}corporis harmoniam dependeat.

\lettrine{\blIniNum{150}}{Future}
autem ex multis \blGetVerse{5}et per alia media concludit idem. \blGetVerse{3}Hic enim primo ostendit, quod anima \blGetVerseX{9}{my note 1 at the margin}rationalis 
non est virtus \blGetVerseX{1}{my note 2 at the margin}in corpore, ita quod secundum esse vel
operationem vel utrumque ad \blGetVerseX{7}{my note 3 at the margin}corporis harmoniam dependeat.

\lettrine{125 }
autem ex multis \blGetVerse{5}et per alia media concludit idem. \blGetVerse{3}Hic enim primo ostendit, quod anima \blGetVerseX{9}{my note 1 at the margin}rationalis 
non est virtus \blGetVerseX{1}{my note 2 at the margin}in corpore, ita quod secundum esse vel
operationem vel utrumque ad \blGetVerseX{7}{my note 3 at the margin}corporis harmoniam dependeat.

\lettrine{\blIniChar{FU}}{\blIniWord{turo}}
autem ex multis \blGetVerse{5}et per alia media concludit idem. \blGetVerse{3}Hic enim primo ostendit, quod anima \blGetVerseX{9}{my note 1 at the margin}rationalis 
non est virtus \blGetVerseX{1}{my note 2 at the margin}in corpore, ita quod secundum esse vel
operationem vel utrumque ad \blGetVerseX{7}{my note 3 at the margin}corporis harmoniam dependeat.

\lettrine{\blIniNum{150}}{Future}
autem ex multis \blGetVerse{5}et per alia media concludit idem. \blGetVerse{3}Hic enim primo ostendit, quod anima \blGetVerseX{9}{my note 1 at the margin}rationalis 
non est virtus \blGetVerseX{1}{my note 2 at the margin}in corpore, ita quod secundum esse vel
operationem vel utrumque ad \blGetVerseX{7}{my note 3 at the margin}corporis harmoniam dependeat.

\lettrine{125 }
autem ex multis \blGetVerse{5}et per alia media concludit idem. \blGetVerse{3}Hic enim primo ostendit, quod anima \blGetVerseX{9}{my note 1 at the margin}rationalis 
non est virtus \blGetVerseX{1}{my note 2 at the margin}in corpore, ita quod secundum esse vel
operationem vel utrumque ad \blGetVerseX{7}{my note 3 at the margin}corporis harmoniam dependeat.

\lettrine{\blIniChar{FU}}{\blIniWord{turo}}
autem ex multis \blGetVerse{5}et per alia media concludit idem. \blGetVerse{3}Hic enim primo ostendit, quod anima \blGetVerseX{9}{my note 1 at the margin}rationalis 
non est virtus \blGetVerseX{1}{my note 2 at the margin}in corpore, ita quod secundum esse vel
operationem vel utrumque ad \blGetVerseX{7}{my note 3 at the margin}corporis harmoniam dependeat.

\lettrine{\blIniNum{150}}{Future}
autem ex multis \blGetVerse{5}et per alia media concludit idem. \blGetVerse{3}Hic enim primo ostendit, quod anima \blGetVerseX{9}{my note 1 at the margin}rationalis 
non est virtus \blGetVerseX{1}{my note 2 at the margin}in corpore, ita quod secundum esse vel
operationem vel utrumque ad \blGetVerseX{7}{my note 3 at the margin}corporis harmoniam dependeat.

\lettrine{125 }
autem ex multis \blGetVerse{5}et per alia media concludit idem. \blGetVerse{3}Hic enim primo ostendit, quod anima \blGetVerseX{9}{my note 1 at the margin}rationalis 
non est virtus \blGetVerseX{1}{my note 2 at the margin}in corpore, ita quod secundum esse vel
operationem vel utrumque ad \blGetVerseX{7}{my note 3 at the margin}corporis harmoniam dependeat.

\lettrine{\blIniChar{FU}}{\blIniWord{turo}}
autem ex multis \blGetVerse{5}et per alia media concludit idem. \blGetVerse{3}Hic enim primo ostendit, quod anima \blGetVerseX{9}{my note 1 at the margin}rationalis 
non est virtus \blGetVerseX{1}{my note 2 at the margin}in corpore, ita quod secundum esse vel
operationem vel utrumque ad \blGetVerseX{7}{my note 3 at the margin}corporis harmoniam dependeat.

\lettrine{\blIniNum{150}}{Future}
autem ex multis \blGetVerse{5}et per alia media concludit idem. \blGetVerse{3}Hic enim primo ostendit, quod anima \blGetVerseX{9}{my note 1 at the margin}rationalis 
non est virtus \blGetVerseX{1}{my note 2 at the margin}in corpore, ita quod secundum esse vel
operationem vel utrumque ad \blGetVerseX{7}{my note 3 at the margin}corporis harmoniam dependeat.

\lettrine{125 }
autem ex multis \blGetVerse{5}et per alia media concludit idem. \blGetVerse{3}Hic enim primo ostendit, quod anima \blGetVerseX{9}{my note 1 at the margin}rationalis 
non est virtus \blGetVerseX{1}{my note 2 at the margin}in corpore, ita quod secundum esse vel
operationem vel utrumque ad \blGetVerseX{7}{my note 3 at the margin}corporis harmoniam dependeat.

\lettrine{\blIniChar{FU}}{\blIniWord{turo}}
autem ex multis \blGetVerse{5}et per alia media concludit idem. \blGetVerse{3}Hic enim primo ostendit, quod anima \blGetVerseX{9}{my note 1 at the margin}rationalis 
non est virtus \blGetVerseX{1}{my note 2 at the margin}in corpore, ita quod secundum esse vel
operationem vel utrumque ad \blGetVerseX{7}{my note 3 at the margin}corporis harmoniam dependeat.

\lettrine{\blIniNum{150}}{Future}
autem ex multis \blGetVerse{5}et per alia media concludit idem. \blGetVerse{3}Hic enim primo ostendit, quod anima \blGetVerseX{9}{my note 1 at the margin}rationalis 
non est virtus \blGetVerseX{1}{my note 2 at the margin}in corpore, ita quod secundum esse vel
operationem vel utrumque ad \blGetVerseX{7}{my note 3 at the margin}corporis harmoniam dependeat.

\lettrine{125 }
autem ex multis \blGetVerse{5}et per alia media concludit idem. \blGetVerse{3}Hic enim primo ostendit, quod anima \blGetVerseX{9}{my note 1 at the margin}rationalis 
non est virtus \blGetVerseX{1}{my note 2 at the margin}in corpore, ita quod secundum esse vel
operationem vel utrumque ad \blGetVerseX{7}{my note 3 at the margin}corporis harmoniam dependeat.

\lettrine{\blIniChar{FU}}{\blIniWord{turo}}
autem ex multis \blGetVerse{5}et per alia media concludit idem. \blGetVerse{3}Hic enim primo ostendit, quod anima \blGetVerseX{9}{my note 1 at the margin}rationalis 
non est virtus \blGetVerseX{1}{my note 2 at the margin}in corpore, ita quod secundum esse vel
operationem vel utrumque ad \blGetVerseX{7}{my note 3 at the margin}corporis harmoniam dependeat.

\lettrine{\blIniNum{150}}{Future}
autem ex multis \blGetVerse{5}et per alia media concludit idem. \blGetVerse{3}Hic enim primo ostendit, quod anima \blGetVerseX{9}{my note 1 at the margin}rationalis 
non est virtus \blGetVerseX{1}{my note 2 at the margin}in corpore, ita quod secundum esse vel
operationem vel utrumque ad \blGetVerseX{7}{my note 3 at the margin}corporis harmoniam dependeat.

\lettrine{125 }
autem ex multis \blGetVerse{5}et per alia media concludit idem. \blGetVerse{3}Hic enim primo ostendit, quod anima \blGetVerseX{9}{my note 1 at the margin}rationalis 
non est virtus \blGetVerseX{1}{my note 2 at the margin}in corpore, ita quod secundum esse vel
operationem vel utrumque ad \blGetVerseX{7}{my note 3 at the margin}corporis harmoniam dependeat.

\lettrine{\blIniChar{FU}}{\blIniWord{turo}}
autem ex multis \blGetVerse{5}et per alia media concludit idem. \blGetVerse{3}Hic enim primo ostendit, quod anima \blGetVerseX{9}{my note 1 at the margin}rationalis 
non est virtus \blGetVerseX{1}{my note 2 at the margin}in corpore, ita quod secundum esse vel
operationem vel utrumque ad \blGetVerseX{7}{my note 3 at the margin}corporis harmoniam dependeat.

\lettrine{\blIniNum{150}}{Future}
autem ex multis \blGetVerse{5}et per alia media concludit idem. \blGetVerse{3}Hic enim primo ostendit, quod anima \blGetVerseX{9}{my note 1 at the margin}rationalis 
non est virtus \blGetVerseX{1}{my note 2 at the margin}in corpore, ita quod secundum esse vel
operationem vel utrumque ad \blGetVerseX{7}{my note 3 at the margin}corporis harmoniam dependeat.

\lettrine{125 }
autem ex multis \blGetVerse{5}et per alia media concludit idem. \blGetVerse{3}Hic enim primo ostendit, quod anima \blGetVerseX{9}{my note 1 at the margin}rationalis 
non est virtus \blGetVerseX{1}{my note 2 at the margin}in corpore, ita quod secundum esse vel
operationem vel utrumque ad \blGetVerseX{7}{my note 3 at the margin}corporis harmoniam dependeat.


\blEndBook



\blSetBookNameMeaning{En el comienzo}
\blSetBookNumber{1}
\blSetHebrewBookName{בראשית}
\blSetBookName{Genesis}
\blSetBookNameTranslit{Bereshit}
\blStartBook

\blSetChapterNumber{1}

%{\blSetAcorn\selectfont ABCDEF GHIJKL MNÑOPQ RSTUVW XYZ 01234 56789}

\lettrine{\blIniChar{FU}}{\blIniWord{turo}}
autem ex multis \blGetVerse{5}et per alia media concludit idem. \blGetVerse{3}Hic enim primo ostendit, quod anima \blGetVerseX{9}{my note 1 at the margin}rationalis 
non est virtus \blGetVerseX{1}{my note 2 at the margin}in corpore, ita quod secundum esse vel
operationem vel utrumque ad \blGetVerseX{7}{my note 3 at the margin}corporis harmoniam dependeat.

\lettrine{\blIniNum{150}}{Future}
autem ex multis \blGetVerse{5}et per alia media concludit idem. \blGetVerse{3}Hic enim primo ostendit, quod anima \blGetVerseX{9}{my note 1 at the margin}rationalis 
non est virtus \blGetVerseX{1}{my note 2 at the margin}in corpore, ita quod secundum esse vel
operationem vel utrumque ad \blGetVerseX{7}{my note 3 at the margin}corporis harmoniam dependeat.

\lettrine{125 }
autem ex multis \blGetVerse{5}et per alia media concludit idem. \blGetVerse{3}Hic enim primo ostendit, quod anima \blGetVerseX{9}{my note 1 at the margin}rationalis 
non est virtus \blGetVerseX{1}{my note 2 at the margin}in corpore, ita quod secundum esse vel
operationem vel utrumque ad \blGetVerseX{7}{my note 3 at the margin}corporis harmoniam dependeat.

\blSubTitle{Subtitulo numero Uno}

\lettrine{\blIniChar{FU}}{\blIniWord{turo}}
autem ex multis \blGetVerse{5}et per alia media concludit idem. \blGetVerse{3}Hic enim primo ostendit, quod anima \blGetVerseX{9}{my note 1 at the margin}rationalis 
non est virtus \blGetVerseX{1}{my note 2 at the margin}in corpore, ita quod secundum esse vel
operationem vel utrumque ad \blGetVerseX{7}{my note 3 at the margin}corporis harmoniam dependeat.

\lettrine{\blIniNum{150}}{Future}
autem ex multis \blGetVerse{5}et per alia media concludit idem. \blGetVerse{3}Hic enim primo ostendit, quod anima \blGetVerseX{9}{my note 1 at the margin}rationalis 
non est virtus \blGetVerseX{1}{my note 2 at the margin}in corpore, ita quod secundum esse vel
operationem vel utrumque ad \blGetVerseX{7}{my note 3 at the margin}corporis harmoniam dependeat.

\lettrine{125 }
autem ex multis \blGetVerse{5}et per alia media concludit idem. \blGetVerse{3}Hic enim primo ostendit, quod anima \blGetVerseX{9}{my note 1 at the margin}rationalis 
non est virtus \blGetVerseX{1}{my note 2 at the margin}in corpore, ita quod secundum esse vel
operationem vel utrumque ad \blGetVerseX{7}{my note 3 at the margin}corporis harmoniam dependeat.

\lettrine{\blIniChar{FU}}{\blIniWord{turo}}
autem ex multis \blGetVerse{5}et per alia media concludit idem. \blGetVerse{3}Hic enim primo ostendit, quod anima \blGetVerseX{9}{my note 1 at the margin}rationalis 
non est virtus \blGetVerseX{1}{my note 2 at the margin}in corpore, ita quod secundum esse vel
operationem vel utrumque ad \blGetVerseX{7}{my note 3 at the margin}corporis harmoniam dependeat.

\lettrine{\blIniNum{150}}{Future}
autem ex multis \blGetVerse{5}et per alia media concludit idem. \blGetVerse{3}Hic enim primo ostendit, quod anima \blGetVerseX{9}{my note 1 at the margin}rationalis 
non est virtus \blGetVerseX{1}{my note 2 at the margin}in corpore, ita quod secundum esse vel
operationem vel utrumque ad \blGetVerseX{7}{my note 3 at the margin}corporis harmoniam dependeat.

\lettrine{125 }
autem ex multis \blGetVerse{5}et per alia media concludit idem. \blGetVerse{3}Hic enim primo ostendit, quod anima \blGetVerseX{9}{my note 1 at the margin}rationalis 
non est virtus \blGetVerseX{1}{my note 2 at the margin}in corpore, ita quod secundum esse vel
operationem vel utrumque ad \blGetVerseX{7}{my note 3 at the margin}corporis harmoniam dependeat.

\lettrine{\blIniChar{FU}}{\blIniWord{turo}}
autem ex multis \blGetVerse{5}et per alia media concludit idem. \blGetVerse{3}Hic enim primo ostendit, quod anima \blGetVerseX{9}{my note 1 at the margin}rationalis 
non est virtus \blGetVerseX{1}{my note 2 at the margin}in corpore, ita quod secundum esse vel
operationem vel utrumque ad \blGetVerseX{7}{my note 3 at the margin}corporis harmoniam dependeat.

\lettrine{\blIniNum{150}}{Future}
autem ex multis \blGetVerse{5}et per alia media concludit idem. \blGetVerse{3}Hic enim primo ostendit, quod anima \blGetVerseX{9}{my note 1 at the margin}rationalis 
non est virtus \blGetVerseX{1}{my note 2 at the margin}in corpore, ita quod secundum esse vel
operationem vel utrumque ad \blGetVerseX{7}{my note 3 at the margin}corporis harmoniam dependeat.

\lettrine{125 }
autem ex multis \blGetVerse{5}et per alia media concludit idem. \blGetVerse{3}Hic enim primo ostendit, quod anima \blGetVerseX{9}{my note 1 at the margin}rationalis 
non est virtus \blGetVerseX{1}{my note 2 at the margin}in corpore, ita quod secundum esse vel
operationem vel utrumque ad \blGetVerseX{7}{my note 3 at the margin}corporis harmoniam dependeat.

\lettrine{\blIniChar{FU}}{\blIniWord{turo}}
autem ex multis \blGetVerse{5}et per alia media concludit idem. \blGetVerse{3}Hic enim primo ostendit, quod anima \blGetVerseX{9}{my note 1 at the margin}rationalis 
non est virtus \blGetVerseX{1}{my note 2 at the margin}in corpore, ita quod secundum esse vel
operationem vel utrumque ad \blGetVerseX{7}{my note 3 at the margin}corporis harmoniam dependeat.

\lettrine{\blIniNum{150}}{Future}
autem ex multis \blGetVerse{5}et per alia media concludit idem. \blGetVerse{3}Hic enim primo ostendit, quod anima \blGetVerseX{9}{my note 1 at the margin}rationalis 
non est virtus \blGetVerseX{1}{my note 2 at the margin}in corpore, ita quod secundum esse vel
operationem vel utrumque ad \blGetVerseX{7}{my note 3 at the margin}corporis harmoniam dependeat.

\lettrine{125 }
autem ex multis \blGetVerse{5}et per alia media concludit idem. \blGetVerse{3}Hic enim primo ostendit, quod anima \blGetVerseX{9}{my note 1 at the margin}rationalis 
non est virtus \blGetVerseX{1}{my note 2 at the margin}in corpore, ita quod secundum esse vel
operationem vel utrumque ad \blGetVerseX{7}{my note 3 at the margin}corporis harmoniam dependeat.

\lettrine{\blIniChar{FU}}{\blIniWord{turo}}
autem ex multis \blGetVerse{5}et per alia media concludit idem. \blGetVerse{3}Hic enim primo ostendit, quod anima \blGetVerseX{9}{my note 1 at the margin}rationalis 
non est virtus \blGetVerseX{1}{my note 2 at the margin}in corpore, ita quod secundum esse vel
operationem vel utrumque ad \blGetVerseX{7}{my note 3 at the margin}corporis harmoniam dependeat.

\lettrine{\blIniNum{150}}{Future}
autem ex multis \blGetVerse{5}et per alia media concludit idem. \blGetVerse{3}Hic enim primo ostendit, quod anima \blGetVerseX{9}{my note 1 at the margin}rationalis 
non est virtus \blGetVerseX{1}{my note 2 at the margin}in corpore, ita quod secundum esse vel
operationem vel utrumque ad \blGetVerseX{7}{my note 3 at the margin}corporis harmoniam dependeat.

\lettrine{125 }
autem ex multis \blGetVerse{5}et per alia media concludit idem. \blGetVerse{3}Hic enim primo ostendit, quod anima \blGetVerseX{9}{my note 1 at the margin}rationalis 
non est virtus \blGetVerseX{1}{my note 2 at the margin}in corpore, ita quod secundum esse vel
operationem vel utrumque ad \blGetVerseX{7}{my note 3 at the margin}corporis harmoniam dependeat.

\lettrine{\blIniChar{FU}}{\blIniWord{turo}}
autem ex multis \blGetVerse{5}et per alia media concludit idem. \blGetVerse{3}Hic enim primo ostendit, quod anima \blGetVerseX{9}{my note 1 at the margin}rationalis 
non est virtus \blGetVerseX{1}{my note 2 at the margin}in corpore, ita quod secundum esse vel
operationem vel utrumque ad \blGetVerseX{7}{my note 3 at the margin}corporis harmoniam dependeat.

\lettrine{\blIniNum{150}}{Future}
autem ex multis \blGetVerse{5}et per alia media concludit idem. \blGetVerse{3}Hic enim primo ostendit, quod anima \blGetVerseX{9}{my note 1 at the margin}rationalis 
non est virtus \blGetVerseX{1}{my note 2 at the margin}in corpore, ita quod secundum esse vel
operationem vel utrumque ad \blGetVerseX{7}{my note 3 at the margin}corporis harmoniam dependeat.

\lettrine{125 }
autem ex multis \blGetVerse{5}et per alia media concludit idem. \blGetVerse{3}Hic enim primo ostendit, quod anima \blGetVerseX{9}{my note 1 at the margin}rationalis 
non est virtus \blGetVerseX{1}{my note 2 at the margin}in corpore, ita quod secundum esse vel
operationem vel utrumque ad \blGetVerseX{7}{my note 3 at the margin}corporis harmoniam dependeat.

\lettrine{\blIniChar{FU}}{\blIniWord{turo}}
autem ex multis \blGetVerse{5}et per alia media concludit idem. \blGetVerse{3}Hic enim primo ostendit, quod anima \blGetVerseX{9}{my note 1 at the margin}rationalis 
non est virtus \blGetVerseX{1}{my note 2 at the margin}in corpore, ita quod secundum esse vel
operationem vel utrumque ad \blGetVerseX{7}{my note 3 at the margin}corporis harmoniam dependeat.

\lettrine{\blIniNum{150}}{Future}
autem ex multis \blGetVerse{5}et per alia media concludit idem. \blGetVerse{3}Hic enim primo ostendit, quod anima \blGetVerseX{9}{my note 1 at the margin}rationalis 
non est virtus \blGetVerseX{1}{my note 2 at the margin}in corpore, ita quod secundum esse vel
operationem vel utrumque ad \blGetVerseX{7}{my note 3 at the margin}corporis harmoniam dependeat.

\lettrine{125 }
autem ex multis \blGetVerse{5}et per alia media concludit idem. \blGetVerse{3}Hic enim primo ostendit, quod anima \blGetVerseX{9}{my note 1 at the margin}rationalis 
non est virtus \blGetVerseX{1}{my note 2 at the margin}in corpore, ita quod secundum esse vel
operationem vel utrumque ad \blGetVerseX{7}{my note 3 at the margin}corporis harmoniam dependeat.

\lettrine{\blIniChar{FU}}{\blIniWord{turo}}
autem ex multis \blGetVerse{5}et per alia media concludit idem. \blGetVerse{3}Hic enim primo ostendit, quod anima \blGetVerseX{9}{my note 1 at the margin}rationalis 
non est virtus \blGetVerseX{1}{my note 2 at the margin}in corpore, ita quod secundum esse vel
operationem vel utrumque ad \blGetVerseX{7}{my note 3 at the margin}corporis harmoniam dependeat.

\lettrine{\blIniNum{150}}{Future}
autem ex multis \blGetVerse{5}et per alia media concludit idem. \blGetVerse{3}Hic enim primo ostendit, quod anima \blGetVerseX{9}{my note 1 at the margin}rationalis 
non est virtus \blGetVerseX{1}{my note 2 at the margin}in corpore, ita quod secundum esse vel
operationem vel utrumque ad \blGetVerseX{7}{my note 3 at the margin}corporis harmoniam dependeat.

\lettrine{125 }
autem ex multis \blGetVerse{5}et per alia media concludit idem. \blGetVerse{3}Hic enim primo ostendit, quod anima \blGetVerseX{9}{my note 1 at the margin}rationalis 
non est virtus \blGetVerseX{1}{my note 2 at the margin}in corpore, ita quod secundum esse vel
operationem vel utrumque ad \blGetVerseX{7}{my note 3 at the margin}corporis harmoniam dependeat.

\lettrine{\blIniChar{FU}}{\blIniWord{turo}}
autem ex multis \blGetVerse{5}et per alia media concludit idem. \blGetVerse{3}Hic enim primo ostendit, quod anima \blGetVerseX{9}{my note 1 at the margin}rationalis 
non est virtus \blGetVerseX{1}{my note 2 at the margin}in corpore, ita quod secundum esse vel
operationem vel utrumque ad \blGetVerseX{7}{my note 3 at the margin}corporis harmoniam dependeat.

\lettrine{\blIniNum{150}}{Future}
autem ex multis \blGetVerse{5}et per alia media concludit idem. \blGetVerse{3}Hic enim primo ostendit, quod anima \blGetVerseX{9}{my note 1 at the margin}rationalis 
non est virtus \blGetVerseX{1}{my note 2 at the margin}in corpore, ita quod secundum esse vel
operationem vel utrumque ad \blGetVerseX{7}{my note 3 at the margin}corporis harmoniam dependeat.

\lettrine{125 }
autem ex multis \blGetVerse{5}et per alia media concludit idem. \blGetVerse{3}Hic enim primo ostendit, quod anima \blGetVerseX{9}{my note 1 at the margin}rationalis 
non est virtus \blGetVerseX{1}{my note 2 at the margin}in corpore, ita quod secundum esse vel
operationem vel utrumque ad \blGetVerseX{7}{my note 3 at the margin}corporis harmoniam dependeat.

\lettrine{\blIniChar{FU}}{\blIniWord{turo}}
autem ex multis \blGetVerse{5}et per alia media concludit idem. \blGetVerse{3}Hic enim primo ostendit, quod anima \blGetVerseX{9}{my note 1 at the margin}rationalis 
non est virtus \blGetVerseX{1}{my note 2 at the margin}in corpore, ita quod secundum esse vel
operationem vel utrumque ad \blGetVerseX{7}{my note 3 at the margin}corporis harmoniam dependeat.

\lettrine{\blIniNum{150}}{Future}
autem ex multis \blGetVerse{5}et per alia media concludit idem. \blGetVerse{3}Hic enim primo ostendit, quod anima \blGetVerseX{9}{my note 1 at the margin}rationalis 
non est virtus \blGetVerseX{1}{my note 2 at the margin}in corpore, ita quod secundum esse vel
operationem vel utrumque ad \blGetVerseX{7}{my note 3 at the margin}corporis harmoniam dependeat.

\lettrine{125 }
autem ex multis \blGetVerse{5}et per alia media concludit idem. \blGetVerse{3}Hic enim primo ostendit, quod anima \blGetVerseX{9}{my note 1 at the margin}rationalis 
non est virtus \blGetVerseX{1}{my note 2 at the margin}in corpore, ita quod secundum esse vel
operationem vel utrumque ad \blGetVerseX{7}{my note 3 at the margin}corporis harmoniam dependeat.

\lettrine{\blIniChar{FU}}{\blIniWord{turo}}
autem ex multis \blGetVerse{5}et per alia media concludit idem. \blGetVerse{3}Hic enim primo ostendit, quod anima \blGetVerseX{9}{my note 1 at the margin}rationalis 
non est virtus \blGetVerseX{1}{my note 2 at the margin}in corpore, ita quod secundum esse vel
operationem vel utrumque ad \blGetVerseX{7}{my note 3 at the margin}corporis harmoniam dependeat.

\lettrine{\blIniNum{150}}{Future}
autem ex multis \blGetVerse{5}et per alia media concludit idem. \blGetVerse{3}Hic enim primo ostendit, quod anima \blGetVerseX{9}{my note 1 at the margin}rationalis 
non est virtus \blGetVerseX{1}{my note 2 at the margin}in corpore, ita quod secundum esse vel
operationem vel utrumque ad \blGetVerseX{7}{my note 3 at the margin}corporis harmoniam dependeat.

\lettrine{125 }
autem ex multis \blGetVerse{5}et per alia media concludit idem. \blGetVerse{3}Hic enim primo ostendit, quod anima \blGetVerseX{9}{my note 1 at the margin}rationalis 
non est virtus \blGetVerseX{1}{my note 2 at the margin}in corpore, ita quod secundum esse vel
operationem vel utrumque ad \blGetVerseX{7}{my note 3 at the margin}corporis harmoniam dependeat.

\lettrine{\blIniChar{FU}}{\blIniWord{turo}}
autem ex multis \blGetVerse{5}et per alia media concludit idem. \blGetVerse{3}Hic enim primo ostendit, quod anima \blGetVerseX{9}{my note 1 at the margin}rationalis 
non est virtus \blGetVerseX{1}{my note 2 at the margin}in corpore, ita quod secundum esse vel
operationem vel utrumque ad \blGetVerseX{7}{my note 3 at the margin}corporis harmoniam dependeat.

\lettrine{\blIniNum{150}}{Future}
autem ex multis \blGetVerse{5}et per alia media concludit idem. \blGetVerse{3}Hic enim primo ostendit, quod anima \blGetVerseX{9}{my note 1 at the margin}rationalis 
non est virtus \blGetVerseX{1}{my note 2 at the margin}in corpore, ita quod secundum esse vel
operationem vel utrumque ad \blGetVerseX{7}{my note 3 at the margin}corporis harmoniam dependeat.

\lettrine{125 }
autem ex multis \blGetVerse{5}et per alia media concludit idem. \blGetVerse{3}Hic enim primo ostendit, quod anima \blGetVerseX{9}{my note 1 at the margin}rationalis 
non est virtus \blGetVerseX{1}{my note 2 at the margin}in corpore, ita quod secundum esse vel
operationem vel utrumque ad \blGetVerseX{7}{my note 3 at the margin}corporis harmoniam dependeat.

\lettrine{\blIniChar{FU}}{\blIniWord{turo}}
autem ex multis \blGetVerse{5}et per alia media concludit idem. \blGetVerse{3}Hic enim primo ostendit, quod anima \blGetVerseX{9}{my note 1 at the margin}rationalis 
non est virtus \blGetVerseX{1}{my note 2 at the margin}in corpore, ita quod secundum esse vel
operationem vel utrumque ad \blGetVerseX{7}{my note 3 at the margin}corporis harmoniam dependeat.

\lettrine{\blIniNum{150}}{Future}
autem ex multis \blGetVerse{5}et per alia media concludit idem. \blGetVerse{3}Hic enim primo ostendit, quod anima \blGetVerseX{9}{my note 1 at the margin}rationalis 
non est virtus \blGetVerseX{1}{my note 2 at the margin}in corpore, ita quod secundum esse vel
operationem vel utrumque ad \blGetVerseX{7}{my note 3 at the margin}corporis harmoniam dependeat.

\lettrine{125 }
autem ex multis \blGetVerse{5}et per alia media concludit idem. \blGetVerse{3}Hic enim primo ostendit, quod anima \blGetVerseX{9}{my note 1 at the margin}rationalis 
non est virtus \blGetVerseX{1}{my note 2 at the margin}in corpore, ita quod secundum esse vel
operationem vel utrumque ad \blGetVerseX{7}{my note 3 at the margin}corporis harmoniam dependeat.

\lettrine{\blIniChar{FU}}{\blIniWord{turo}}
autem ex multis \blGetVerse{5}et per alia media concludit idem. \blGetVerse{3}Hic enim primo ostendit, quod anima \blGetVerseX{9}{my note 1 at the margin}rationalis 
non est virtus \blGetVerseX{1}{my note 2 at the margin}in corpore, ita quod secundum esse vel
operationem vel utrumque ad \blGetVerseX{7}{my note 3 at the margin}corporis harmoniam dependeat.

\lettrine{\blIniNum{150}}{Future}
autem ex multis \blGetVerse{5}et per alia media concludit idem. \blGetVerse{3}Hic enim primo ostendit, quod anima \blGetVerseX{9}{my note 1 at the margin}rationalis 
non est virtus \blGetVerseX{1}{my note 2 at the margin}in corpore, ita quod secundum esse vel
operationem vel utrumque ad \blGetVerseX{7}{my note 3 at the margin}corporis harmoniam dependeat.

\lettrine{125 }
autem ex multis \blGetVerse{5}et per alia media concludit idem. \blGetVerse{3}Hic enim primo ostendit, quod anima \blGetVerseX{9}{my note 1 at the margin}rationalis 
non est virtus \blGetVerseX{1}{my note 2 at the margin}in corpore, ita quod secundum esse vel
operationem vel utrumque ad \blGetVerseX{7}{my note 3 at the margin}corporis harmoniam dependeat.

\lettrine{\blIniChar{FU}}{\blIniWord{turo}}
autem ex multis \blGetVerse{5}et per alia media concludit idem. \blGetVerse{3}Hic enim primo ostendit, quod anima \blGetVerseX{9}{my note 1 at the margin}rationalis 
non est virtus \blGetVerseX{1}{my note 2 at the margin}in corpore, ita quod secundum esse vel
operationem vel utrumque ad \blGetVerseX{7}{my note 3 at the margin}corporis harmoniam dependeat.

\lettrine{\blIniNum{150}}{Future}
autem ex multis \blGetVerse{5}et per alia media concludit idem. \blGetVerse{3}Hic enim primo ostendit, quod anima \blGetVerseX{9}{my note 1 at the margin}rationalis 
non est virtus \blGetVerseX{1}{my note 2 at the margin}in corpore, ita quod secundum esse vel
operationem vel utrumque ad \blGetVerseX{7}{my note 3 at the margin}corporis harmoniam dependeat.

\lettrine{125 }
autem ex multis \blGetVerse{5}et per alia media concludit idem. \blGetVerse{3}Hic enim primo ostendit, quod anima \blGetVerseX{9}{my note 1 at the margin}rationalis 
non est virtus \blGetVerseX{1}{my note 2 at the margin}in corpore, ita quod secundum esse vel
operationem vel utrumque ad \blGetVerseX{7}{my note 3 at the margin}corporis harmoniam dependeat.


\blEndBook


%\end{multicols}
\end{document}

