\blSetBookName{La Biblia Cronológica}
\blSetBookNameTranslit{Hataniikh Hacronologi}
\blSetBookNameMeaning{}
\blSetHebrewBookName{התנייך הכרונולוגי}
\blSetBookNumber{1}
\blStartBook

\blSubTitle{Antes de la creación}

considerando que habéis sido rescatados de vuestro vano vivir según la tradición de vuestros padres, no con plata y oro, corruptibles, \blGetCronoVerse{1Pe.1:18}

sino con la sangre preciosa de Cristo, como cordero sin defecto ni mancha, \blGetCronoVerse{1Pe.1:19}

ya conocido antes de la creación del mundo y manifestado al fin de los tiempos por amor vuestro;"\blGetCronoVerse{1Pe.1:20}

para que la multiforme sabiduría de Dios sea ahora notificada por medio de la Iglesia a los principados y potestades en los cielos, \blGetCronoVerse{Efe.3:10}

conforme al plan eterno que El ha realizado en Cristo Jesús, nuestro Señor, \blGetCronoVerse{Efe.3:11}

Tú, ¡oh Yahvé! Dios mío, has multiplicado tus maravillas y tus designios en favor nuestro. Nadie hay semejante a ti. Yo quisiera anunciarlas, hablar de ellas, pero sobrepasan todo número. \blGetCronoVerse{Sal.40:6}

\textbf{Sal 40:7} No te complaces tú en el sacrificio y la ofrenda; me has dado oído abierto; no pides ni holocausto ni sacrificio expiatorio." 

\textbf{Sal 40:8} Entonces dije: “¡He aquí que vengo!” En el rollo del libro me está prescrito 

\textbf{Heb 10:5} Por lo cual, entrando en este mundo, dice: “No quisiste sacrificios ni oblaciones, pero me has preparado un cuerpo. 

\textbf{Heb 10:6} Los holocaustos y sacrificios por el pecado no los recibiste. 

\textbf{Heb 10:7} Entonces dije: He aquí que vengo — en el volumen del libro está escrito de mí — para hacer,oh Dios!, tu voluntad.” 

\textbf{Heb 10:8} Habiendo dicho arriba: “Los sacrificios, las ofrendas y los holocaustos por el pecado no los quieres, no los aceptas,” siendo todos ofrecidos según la Ley, 

\textbf{Heb 10:9} dijo entonces: “He aquí que vengo para hacer tu voluntad.” Abroga lo primero para establecer lo segundo. 

\textbf{Heb 10:10} En virtud de esta voluntad somos nosotros santificados por la oblación del cuerpo de Jesucristo, hecha una sola vez. 



\blSubTitle{El Verbo hecho carne}

\textbf{Jua 1:1} Al principio era el Verbo, y el Verbo estaba en Dios, y el Verbo era Dios. 

\textbf{Jua 1:2} El estaba al principio en Dios. 

\textbf{Jua 1:3} Todas las cosas fueron hechas por EL, y sin El no se hizo nada de cuanto ha sido hecho. 

\textbf{1Co 8:6} para nosotros no hay más que un Dios, el Padre, de quien todo procede y para quien somos nosotros, y un solo Señor, Jesucristo, por quien son todas las cosas y nosotros también. 

\textbf{Col 1:16} porque en El fueron creadas todas las cosas del cielo y de la tierra, las visibles y las invisibles, los tronos, las dominaciones, los principados, las potestades; todo fue creado por El y para El." 

\textbf{Col 1:17} El es antes que todo, y todo subsiste en él. 

\textbf{Hch 17:24} El Dios que hizo el mundo y todas las cosas que hay en él, ése, siendo Señor del cielo y de la tierra, no habita en templos hechos por mano de hombre, 

\textbf{Hch 17:25} ni por manos humanas es servido, como si necesitase de algo, siendo El mismo quien da a todos la vida, el aliento y todas las cosas. 

\textbf{Hch 17:26} El hizo de uno todo el linaje humano, para poblar toda la haz de la tierra; El fijó a los pueblos los tiempos establecidos y los límites de su habitación," 

\textbf{Hch 17:27} para que busquen a Dios y siquiera a tientas le hallen, que no está lejos de nosotros, 

\textbf{Hch 17:28} porque en El vivimos y nos movemos y existimos, como algunos de vuestros poetas han dicho: “porque somos linaje suyo.” 

\textbf{Hch 17:29} Siendo, pues, linaje de Dios, no debemos pensar que la divinidad es semejante al oro o a la plata o a la piedra, obra del arte y del pensamiento humano. 



\blSubTitle{La fe}

\textbf{Heb 11:1} Ahora bien, la fe es garantía de lo que esperamos, prueba de lo que no vemos;" 

\textbf{Heb 11:2} pues por ella adquirieron gran nombre los antiguos. 

\textbf{Heb 11:3} Por la fe conocemos que los mundos han sido dispuestos por la palabra de Dios, de suerte que de lo invisible ha tenido origen lo visible. 

\textbf{Heb 11:4} Por la fe, Abel ofreció a Dios sacrificios más excelentes que Caín y por ellos fue declarado justo, dando Dios testimonio a sus ofrendas; y por ella habla aun después de muerto*" 

\textbf{Heb 11:5} Por la fe fue trasladado Enoc, sin pasar por la muerte, y no fue hallado, porque Dios le trasladó. Pero antes de ser trasladado recibió el testimonio de haber agradado a Dios, 

\textbf{Heb 11:6} cosa que sin la fe es imposible. Que es preciso que quien se acerque a Dios crea que existe y que es remunerador de los que le buscan. 

\textbf{Heb 11:7} Por la fe, Noé, avisado por divina revelación de lo que aún no se veía, movido de temor, fabricó el arca para salvación de su casa; y por aquella misma fe condenó al mundo, haciéndose heredero de la justicia según la fe." 

\textbf{Heb 11:8} Por la fe, Abraham, al ser llamado, obedeció y salió hacia la tierra que había de recibir en herencia, pero sin saber adonde iba. 

\textbf{Heb 11:9} Por la fe moró en la tierra de sus promesas como en tierra extraña, habitando en tiendas, lo mismo que Isaac y Jacob, coherederos de la misma promesa. 

\textbf{Heb 11:10} Porque esperaba él ciudad asentada sobre firmes cimientos, cuyo arquitecto y constructor sería Dios. 

\textbf{Heb 11:11} Por la fe, la misma Sara recibió el vigor, principio de una descendencia, y esto fuera ya de la edad propicia, por cuanto creyó que era fiel el que se lo había prometido. 

\textbf{Heb 11:12} Y por eso de uno, y éste ya sin vigor para engendrar, nacieron hijos numerosos como las estrellas del cielo y como las arenas incontables que hay en las riberas del mar. 

\textbf{Heb 11:13} En la fe murieron todos sin recibir las promesas; pero viéndolas de lejos y saludándolas y confesándose peregrinos y huéspedes sobre la tierra," 

\textbf{Heb 11:14} pues los que tales cosas dicen dan bien a entender que buscan la patria. 

\textbf{Heb 11:15} Que si se acordaran de aquélla de donde habían salido, tiempo tuvieron para volverse a ella. 

\textbf{Heb 11:16} Pero deseaban otra mejor, esto es, la celestial. Por eso Dios no se avergüenza de llamarse Dios suyo, porque les tenía preparada una ciudad. 

\textbf{Heb 11:17} Por la fe ofreció Abraham a Isaac cuando fue puesto a prueba, y ofreció a su unigénito, el que había recibido las promesas, 

\textbf{Heb 11:18} y de quien se había dicho: “Por Isaac tendrás tu descendencia”;" 

\textbf{Heb 11:19} pensando que hasta de entre los muertos podría Dios resucitarle. Por eso le recuperó también en figura. 

\textbf{Heb 11:20} Por la fe dio Isaac las bendiciones de los bienes futuros a Jacob y a Esaú. 

\textbf{Heb 11:21} Por la fe, Jacob, moribundo, bendijo a cada uno de los hijos de José, apoyándose en la extremidad de su báculo. 

\textbf{Heb 11:22} Por la fe, José, estando para acabar, se acordó de la salida de los hijos de Israel y dio órdenes acerca de sus huesos. 



\blSubTitle{La creación}

\textbf{Gen 1:1} Al principio creó Dios los cielos y la tierra. 

\textbf{Gen 1:2} La tierra estaba confusa y vacía, y las tinieblas cubrían la haz del abismo, pero el espíritu de Dios se cernía sobre la superficie de las aguas. 

\textbf{Gen 1:3} Dijo Dios: “Haya luz,” y hubo luz. 

\textbf{Gen 1:4} y vio Dios ser buena la luz, y la separó de las tinieblas; 

\textbf{Gen 1:5} y a la luz llamó día y a las tinieblas noche, y hubo tarde y mañana, día primero. 

\textbf{Gen 1:6} Dijo luego Dios: “Haya firmamento en medio de las aguas, que separe unas de otras”; y así fue. 

\textbf{Gen 1:7} E hizo Dios el firmamento, separando aguas de aguas, las aguas que estaban debajo del firmamento de las que estaban sobre el firmamento. Y vio Dios ser bueno. 

\textbf{Gen 1:8} Llamó Dios al firmamento cielo, y hubo tarde y mañana, día segundo. 

\textbf{Gen 1:9} Dijo luego: “Júntense en un lugar las aguas de debajo de los cielos y aparezca lo seco.” Así se hizo, 

\textbf{Gen 1:10} y se juntaron las aguas de debajo de los cielos en sus lugares y apareció lo seco; y a lo seco llamó Dios tierra, y a la reunión de las aguas, mares. Y vio Dios ser bueno. 

\textbf{Gen 1:11} Dijo luego: “Haga brotar la tierra hierba verde, hierba con semilla y árboles frutales, cada uno con su fruto según su especie y con su simiente, sobre la tierra.” Y así fue. 

\textbf{Gen 1:12} Y produjo la tierra hierba verde, hierba con semilla, y árboles frutales, con su semilla cada uno. Vio Dios ser bueno; 

\textbf{Gen 1:13} y hubo tarde y mañana, día tercero. 

\textbf{Gen 1:14} Dijo luego Dios: “Haya en el firmamento de los cielos lumbreras para separar el día de la noche y servir de señales a estaciones, días y años;" 

\textbf{Gen 1:15} y luzcan en el firmamento de los cielos, para alumbrar la tierra.” Y así fue. 

\textbf{Gen 1:16} Hizo Dios los dos grandes luminares, el mayor para presidir el día, y el menor para presidir la noche, y las estrellas; 

\textbf{Gen 1:17} y los puso en el firmamento de los cielos para alumbrar la tierra, 

\textbf{Gen 1:18} y presidir el día y la noche, y separar la luz de las tinieblas. Y vio Dios ser bueno, 

\textbf{Gen 1:19} y hubo tarde y mañana, día cuarto. 

\textbf{Gen 1:20} Dijo luego Dios: “Hiervan de animales las aguas y vuelen sobre la tierra las aves bajo el firmamento de los cielos.” Y así fue. 

\textbf{Gen 1:21} Y creó Dios los grandes monstruos del agua y todos los animales que bullen en ella, según su especie, y todas las aves aladas, según su especie. Y vio Dios ser bueno, 

\textbf{Gen 1:22} y los bendijo diciendo: “Procread y multiplicaos, y henchid las aguas del mar, y multiplíquense sobre la tierra las aves.” 

\textbf{Gen 1:23} Y hubo tarde y mañana, día quinto. 

\textbf{Gen 1:24} Dijo luego Dios: “Brote la tierra seres animados según su especie, ganados, reptiles, bestias de la tierra según su especie.” Y así fue. 

\textbf{Gen 1:25} Hizo Dios todas las bestias de la tierra según su especie, los ganados según su especie y todos los reptiles de la tierra según su especie. Y vio Dios ser bueno. 

\textbf{Gen 1:26} Díjose entonces Dios: “Hagamos al hombre a nuestra imagen y semejanza, para que domine sobre los peces del mar, sobre las aves del cielo, sobre los ganados y sobre las bestias de la tierra, y sobre cuantos animales se mueven sobre ella.” 

\textbf{Gen 1:27} Y creó Dios al hombre a imagen suya, a imagen de Dios le creó, y los creó macho y hembra; 

\textbf{Gen 1:28} y los bendijo Dios, diciéndoles: “Procread y multiplicaos, y henchid la tierra; sometedla y dominad sobre los peces del mar, sobre las aves del cielo y sobre los ganados, y sobre todo cuanto vive y se mueve sobre la tierra.” 

\textbf{Gen 1:29} Dijo también Dios: “Ahí os doy cuantas hierbas de semilla hay sobre la haz dé la tierra, y cuantos árboles producen fruto de simiente, para que todos os sirvan de alimento. 

\textbf{Gen 1:30} También a todos los animales de la tierra, y a todas las aves del cielo, y a todos los vivientes que sobre la tierra están y se mueven, les doy por comida cuanto de verde hierba la tierra produce.” Y así fue. 

\textbf{Gen 1:31} Y vio Dios ser muy bueno cuanto había hecho, y hubo tarde y mañana, día sexto. 

\textbf{Exo 20:11} pues en seis días hizo Yahvé los cielos y la tierra, el mar y cuanto en ellos se contiene, y el séptimo descansó; por eso bendijo Yahvé el día del sábado y lo santificó." 



\blSubTitle{El hombre en el huerto del Edén}

\textbf{Gen 2:4} Este es el origen de los cielos y la tierra cuando fueron creados. Al tiempo de hacer Yahvé Elohim los cielos y la tierra, 

\textbf{Gen 2:5} no había aún arbusto alguno en el campo, ni germinaba la tierra hierbas, por no haber todavía llovido Yahvé Elohim sobre la tierra ni haber todavía hombre que la labrase, 

\textbf{Gen 2:6} y sacase agua de la tierra para regar toda la superficie del suelo. 

\textbf{Gen 2:7} Formó Yahvé Elohim al hombre del polvo de la tierra y le inspiró en el rostro aliento de vida, y fue así el hombre ser animado. 

\textbf{Gen 2:8} Plantó luego Yahvé Elohim un jardín en Edén, al oriente, y allí puso al hombre a quien formara” 

\textbf{Gen 2:9} Hizo Yahvé Elohim brotar en él de la tierra toda clase de árboles hermosos a la vista y sabrosos al paladar, y en el medio del jardín el árbol de la vida y el árbol de la ciencia del bien y del mal. 

\textbf{Gen 2:10} Salía de Edén un río qué regaba el jardín, y de allí se partía en cuatro brazos. 

\textbf{Gen 2:11} El primero se llamaba Pisón, y es el que rodea toda la tierra de Evila, donde abunda el oro, 

\textbf{Gen 2:12} un oro muy fino, y a más también bedelio y ágata; 

\textbf{Gen 2:13} y el segundo se llama Guijón, y es el que rodea toda la tierra de Cus;" 

\textbf{Gen 2:14} el tercero se llama Tigris, y corre al oriente de Asiria; el cuarto es el Eufrates." 

\textbf{Gen 2:15} Tomó, pues, Yahvé Elohim al hombre y le puso en el jardín de Edén para que lo cultivase y guardase, 

\textbf{Gen 2:16} y le dio este mandato: “De todos los árboles del paraíso puedes comer, 

\textbf{Gen 2:17} pero del árbol de la ciencia del bien y del mal no comas, porque el día que de él comieres, ciertamente morirás.” 

\textbf{Gen 2:18} Y se dijo Yahvé Elohim: “No es bueno que el hombre esté solo; voy a hacerle una ayuda semejante a él.” 

\textbf{Gen 2:19} Y Yahvé Elohim trajo ante Adán todos cuantos animales del campa y cuantas aves del cielo formó de la tierra, para que viese cómo los llamaría, y fuese el nombre de todos los vivientes el que él les diera. 

\textbf{Gen 2:20} Y dio Adán nombre a todos los ganados, y a todas las aves del cielo, y a todas las bestias del campo; pero entre todos ellos no había paía Adán ayuda, semejante a él." 

\textbf{Gen 2:21} Hizo, pues, Yahvé Elohim caer sobre Adán un profundó sopor, y, dormido, tomó una de sus costillas, cerrando en su lugar la carne, 

\textbf{Gen 2:22} y de la costilla que de Adán tomara, formó Yahvé Dios a la mujer y se la presentó a Adán. 

\textbf{Gen 2:23} Adán exclamó: “Esto sí que es ya hueso de mis huesos y carne de mi carne.” Esto se llamará varona, porque del varón ha sido tomada. 

\textbf{Gen 2:24} Dejará el hombre a su padre y a su madre y se adherirá a su mujer, y vendrán a ser los dos una sola carne.” 

\textbf{Gen 2:25} Estaban ambos desnudos, Adán y su mujer, sin avergonzarse de ello. 

\textbf{Mat 19:4} El respondió: ¿No habéis leído que al principio el Creador los hizo varón y hembra? 

\textbf{Mat 19:5} Dijo: “Por esto dejará el hombre al padre y a la madre y se unirá a la mujer, y serán los dos una sola carne.” 

\textbf{Mat 19:6} De manera que ya no son dos, sino una sola carne. Por tanto, lo que Dios unió no lo separe el hombre. 

\textbf{Gen 2:1} Así fueron acabados los cielos, y la tierra, y todo su cortejo. 

\textbf{Gen 2:2} Y, rematada en el día sexto toda la obra que había hecho, descansó Dios el día séptimo de cuanto hiciera; 

\textbf{Gen 2:3} y bendijo el día séptimo y lo santificó, porque en él descansó Dios de cuanto había creado y hecho. 

\textbf{Isa 14:4} entonarás esta sátira contra el rey de Babilonia, y dirás: 

\textbf{Isa 14:5} Rompió Yahvé la vara de los impíos, el cetro de los tiranos. 

\textbf{Isa 14:6} El que castigaba los pueblos con furor, sin cansarse de fustigar; el que en su cólera subyugaba las naciones bajo un yugo sin piedad." 

\textbf{Isa 14:7} Toda la tierra está en paz, toda en reposo, exulta de alegría. 


\nopagebreak

\blEndBook


