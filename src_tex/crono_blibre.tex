
\documentclass[twoside,12pt]{book}
\usepackage{polyglossia}
\usepackage{fontspec}
\usepackage{xunicode}
\usepackage{fancyhdr}
\usepackage{lettrine}
\usepackage{microtype}
\usepackage{color}
\usepackage{xcolor}
\usepackage{geometry}
%\usepackage[showframe]{geometry}
%\usepackage[para,symbol*,bottom,flushmargin]{footmisc}
%\usepackage[para,symbol*,bottom]{footmisc}
%\usepackage[para,symbol*]{footmisc}
\usepackage[para,symbol*,flushmargin]{footmisc}
\usepackage{fourier-orns}
\usepackage{graphicx}
\usepackage{booktabs}
\usepackage{multicol}
\usepackage[object=vectorian]{pgfornament}
\usepackage[explicit]{titlesec}
\usepackage{anyfontsize}
%\usepackage[dolayout]{xdoc2}
%\usepackage{xdoc2}

\renewcommand{\familydefault}{pplj} 
\setdefaultlanguage{english}
\setotherlanguage{hebrew}

\setlength{\parindent}{0cm}

%\renewcommand{\familydefault}{\sfdefault}
\renewcommand{\familydefault}{\rmdefault}

%\titlespacing{\paragraph}{0pt}{0pt}{.5em}[]
\titlespacing{\paragraph}{0pt}{0pt}{.1em}[]

% \raggedbottom

% Footnote settings
%----------------------------------------------------

% The prefix 'bl' is used for all defined vars, commands, etc. meaning 'biblia-libre'

% Using the same symbol for all footnotes because:
%	1. many footnotes (like in a bible) always crash footnote counter.
%	2. footnote counter is reset every footnote.
%	3. footnotes must not use numbers to avoid confusing them with verse numbers.

\DefineFNsymbols*{blSymbols}{*****************************************************************
**********************************************************************************************}

\setfnsymbol{blSymbols}

% \interfootnotelinepenalty=0
% \addtolength{\skip\footins}{2pc plus 5pt}
% \setlength{\skip\footins}{5pt}

% Page layout.
%-----------------------------------------------------

\geometry{
	paper = letterpaper,
	layout = letterpaper,
	top = 20mm,
	headsep = 2mm,
	headheight = 17mm,
	bottom = 20mm,
	footskip = 2mm,
	inner = 25mm,
	outer = 10mm,
}
%	inner = 25mm,
%	outer = 10mm,

%\pagenumbering{gobble}

% Verse indentation.
%-----------------------------------------------------

\newenvironment{verseindent}[1]
  {\begin{list}
          {}
          {\setlength{\itemindent}{-#1}%%'
           \setlength{\leftmargin}{#1}%%'
           \setlength{\itemsep}{0pt}%%'
           \setlength{\parsep}{\parskip}%%'
           \setlength{\topsep}{\parskip}%%'
           }
    \setlength{\parindent}{-#1}%%
    \item[]
  }
  {\end{list}}

% Fonts to use. 
%-----------------------------------------------------

% in fonts folder
\newfontface\blSetDSHermannGotisch{DSHermannGotisch}
\newfontface\blSetDSRomantiques{DSRomantiques}
\newfontface\blSetWieynkFrakturVignetten{WieynkFrakturVignetten}
\newfontface\blSetRothenburgDecorative{RothenburgDecorative}

% in texlive full install
\input Acorn.fd
\newfontface\blSetAcorn{Acorn}

% in system (linux)
\newfontface\blSetDavidCLM{David CLM}

\newcommand{\blHebrewOne}[2]{\blSetDavidCLM\fontsize{#1}{40}\selectfont \RL{#2}}

% Some colors to use.
%-----------------------------------------------------
\definecolor{blMarron}{RGB}{60,30,10}
\definecolor{blDarkblue}{RGB}{0,0,80}
\definecolor{blLightblue}{RGB}{80,80,80}
\definecolor{blDarkgreen}{RGB}{0,80,0}
\definecolor{blDarkgray}{RGB}{0, 51, 153}
\definecolor{blDarkred}{RGB}{80,0,0}

\definecolor{blLayoutColor}{RGB}{0, 51, 153}
% \definecolor{blLayoutColor}{blue}{0.6}
\colorlet{blNormalColor}{black}

% Header and Footer variables.
%-----------------------------------------------------

\newcommand{\blBookNameMeaning}{}
\newcommand{\blBookNumber}{MyBookNumber}
\newcommand{\blHebrewBookName}{MyHebrewBookName}
\newcommand{\blBookName}{MyBookName}
\newcommand{\blBookNameTranslit}{MyBookNameTranslit}

\newcommand{\blChapterNumber}{}

% Variable setters y formated getters
%-----------------------------------------------------

\newcommand{\blSetBookNameMeaning}[1]{\renewcommand{\blBookNameMeaning}{#1}}
\newcommand{\blSetBookNumber}[1]{\renewcommand{\blBookNumber}{#1}}
\newcommand{\blSetHebrewBookName}[1]{\renewcommand{\blHebrewBookName}{#1}}
\newcommand{\blSetBookName}[1]{\renewcommand{\blBookName}{#1}}
\newcommand{\blSetBookNameTranslit}[1]{\renewcommand{\blBookNameTranslit}{#1}}

\newcommand{\blSetChapterNumber}[1]{\renewcommand{\blChapterNumber}{#1}}

\newcommand{\blGetVerse}[1]{%
\setcounter{footnote}{0}%
\textsuperscript{#1}%
}

\newcommand{\blGetVerseX}[2]{%
\footnote{#2}\blGetVerse{#1}%
}

\newcommand{\blIniChar}[1]{%
\blSetAcorn\selectfont{#1}%
}

\newcommand{\blIniWord}[1]{%
\blSetRothenburgDecorative\selectfont{#1}%
}

\newcommand{\blIniNum}[1]{%
\blSetDSRomantiques\selectfont{#1}%
}

\newcommand{\blGetCronoVerse}[1]{%
{\fontsize{6}{8}\selectfont #1}%
}


% Header and Footer look.
%-----------------------------------------------------

\newcommand{\blOrnament}{ \vspace{3ex}\noindent \textcolor{blLayoutColor}{\hrulefill~ \raisebox{-2.5pt}[10pt][10pt]{\leafright \decofourleft \decothreeleft  \aldineright \decotwo \floweroneleft \decoone   \floweroneright \decotwo \aldineleft\decothreeright \decofourright \leafleft} ~  \hrulefill \\ \vspace{3ex}}}

\newcommand{\blOddOrnament}{\noindent \textcolor{blLayoutColor}{ \raisebox{-1.9pt}[10pt][10pt]{\leafright} \hrulefill \raisebox{-1.9pt}[10pt][10pt]{\leafright \decofourleft \decothreeleft  \aldineright \decotwo \floweroneleft \decoone}}}

\newcommand{\blEvenOrnament}{\noindent \textcolor{blLayoutColor}{\raisebox{-1.9pt}[10pt][10pt]{\decoone \floweroneright \decotwo \aldineleft \decothreeright \decofourright \leafleft} \hrulefill \raisebox{-1.9pt}[10pt][10pt]{\leafleft}}}

\newcommand{\blTopOrnament}{\noindent \textcolor{blLayoutColor}{\raisebox{-2.1pt}[10pt][10pt]{\leafright} \hrulefill \raisebox{-2.1pt}[10pt][10pt]{~~~\decofourleft \decotwo \decofourright~~~} \hrulefill \raisebox{-2.1pt}[10pt][10pt]{ \leafleft}} \vspace{0mm}}

\newcommand\blGetOrnament[3][black]{
	\resizebox{#2}{!}{\pgfornament[color = #1,ydelta=-40pt]{#3}}
}

\newcommand\blBookOrnament{{\blSetWieynkFrakturVignetten\fontsize{50}{40}\selectfont P}}

\newcommand\blGetBookHebrewTitle{\hfill \hfill {\blHebrewOne{50}{\blHebrewBookName}} \hfill \blBookOrnament}

% \blGetOrnament[blLayoutColor]{0.7\linewidth}{85} \\
% \blGetOrnament[blLayoutColor]{0.2\linewidth}{71} \\

\newcommand\blGetBookTitle{
	\centerline{\textcolor{blLayoutColor}{\blGetBookHebrewTitle}}
	\centerline{\textcolor{blLayoutColor}{\blBookNameTranslit}}
	\centerline{\textcolor{blLayoutColor}{\blBookNameMeaning}}
	\centerline{\textcolor{blLayoutColor}{(\blBookName)}}
}


\makeatletter
\def\headrule{\blTopOrnament}
\makeatother

\pagestyle{fancy}

\fancyhf{}

\newcommand{\blTopInner}{\textcolor{blLayoutColor}{\blBookName}}
\newcommand{\blTopCenter}{\textcolor{blLayoutColor}{\thepage}}
\newcommand{\blTopOuter}{\textcolor{blLayoutColor}{\blBookNameTranslit}}

\newcommand{\blBottomOuter}{\textcolor{blLayoutColor}{\thepage} \space \blHebrewOne{12}{\blHebrewBookName}}
\newcommand{\blBottomOdd}{
	\blEvenOrnament \\ \large \hfill \sffamily\bf \textcolor{blLayoutColor}{\leafNE ~~~ \blBottomOuter}
}
\newcommand{\blBottomEven}{
	\blOddOrnament \\ \large \sffamily\bf \textcolor{blLayoutColor}{\blBottomOuter ~~~ \reflectbox{\leafNE}} \hfill
}

\fancyhead[LO,RE]{\blTopInner}
\fancyhead[CO,CE]{\blTopCenter}
\fancyhead[RO,LE]{\blTopOuter}

\fancyfoot[LO]{\blBottomOdd}
\fancyfoot[RE]{\blBottomEven}

\fancypagestyle{plain}{
	\fancyhead{}
	\renewcommand{\headrulewidth}{0pt}
	\renewcommand{\headrule}{}
}

\newcommand{\blStartBook}{%
	\thispagestyle{plain}%
	\blGetBookTitle%
}

\newcommand{\blEndBook}{%
	\vspace{1em}
	\centerline{\textcolor{blLayoutColor}{\reflectbox{\blBookOrnament}\blBookOrnament}}%
	\cleardoublepage%
}

\newcommand{\blSubTitle}[1]{
	\vspace{1em}
	%\centerline{\textcolor{blLayoutColor}{\textbf{\fontsize{10}{3}\selectfont #1}}}%
	\centerline{\textcolor{blLayoutColor}{\textbf{\large #1 } \small}}\nopagebreak%
}


%====================================================================================================
%START OF DOCUMENT
%====================================================================================================
\begin{document}

\fontsize{12}{14}\selectfont

%\begin{verseindent}{3em}

\input{books/book_crono_bible.tex}
%\blSetBookName{La Biblia Cronológica}
\blSetBookNameTranslit{Hataniikh Hacronologi}
\blSetBookNameMeaning{}
\blSetHebrewBookName{התנייך הכרונולוגי}
\blSetBookNumber{1}
\blStartBook


\blSubTitle{Antes de la creación}

\textbf{1Pe 1:18} considerando que habéis sido rescatados de vuestro vano vivir según la tradición de vuestros padres, no con plata y oro, corruptibles, 

\textbf{1Pe 1:19} sino con la sangre preciosa de Cristo, como cordero sin defecto ni mancha, 

\textbf{1Pe 1:20} ya conocido antes de la creación del mundo y manifestado al fin de los tiempos por amor vuestro;" 

\textbf{Efe 3:10} para que la multiforme sabiduría de Dios sea ahora notificada por medio de la Iglesia a los principados y potestades en los cielos, 

\textbf{Efe 3:11} conforme al plan eterno que El ha realizado en Cristo Jesús, nuestro Señor, 


\textbf{Rev 17:17} Porque Dios puso en su corazón ejecutar su designio, un solo designio, y dar a la bestia la soberanía sobre ella, hasta que se cumplan las palabras de Dios. 

\textbf{Rev 17:18} La mujer que has visto es aquella ciudad grande que tiene la soberanía sobre todos los reyes de la tierra. 

\textbf{Rev 18:1} Después de estas cosas vi otro ángel que bajaba del cielo con gran poder, a cuya claridad quedó la tierra iluminada. 

\textbf{Rev 18:2} Gritó con poderosa voz, diciendo: Cayó, cayó la gran Babilonia, y quedó convertida en morada de demonios, y guarida de todo espíritu inmundo, y albergue de toda ave inmunda y abominable;" 

\textbf{Rev 18:3} porque del vino de la cólera de su fornicación bebieron todas las naciones, y con ella fornicaron los reyes de la tierra, y los comerciantes de toda la tierra con el poder de su lujo se enriquecieron. 

\textbf{Rev 18:4} Oí otra voz del ciclo que decía: Sal de ella, pueblo mío, para que no os contaminéis con sus pecados y para que no os alcance parte de sus plagas;" 

\textbf{Rev 18:5} porque sus pecados se amontonaron hasta llegar al cielo, y Dios se acordó de sus iniquidades. 

\textbf{Rev 18:6} Dadle según lo que ella dio, y dadle el doble de sus obras; en la copa en que ella mezcló, mezcladle al doble;" 

\textbf{Rev 18:7} cuanto se envaneció y entregó al lujo, dadle otro tanto de tormento y duelo. Ya que dijo en su corazón: Como reina estoy sentada, yo no soy viuda ni veré duelo jamás;" 

\textbf{Rev 18:8} por eso vendrán en un día sus plagas, la mortandad, el duelo y el hambre, y será consumida por el fuego, pues poderoso es el Señor Dios que la ha juzgado. 

\textbf{Rev 18:9} Llorarán, y por ella se herirán los reyes de la tierra que con ella fornicaban y se entregaban al lujo, cuando vean el humo de su incendio, 

\textbf{Rev 18:10} y se detendrán a lo lejos por el temor de su tormento, diciendo: ¡Ay, ay de la ciudad grande, de Babilonia, la ciudad fuerte, porque en una hora ha venido su juicio! 

\textbf{Rev 18:11} Llorarán y se lamentarán los mercaderes de la tierra por ella, porque no hay quien compre sus mercaderías, 

\textbf{Rev 18:12} las mercaderías de oro, de plata, de piedras preciosas, de perlas, de lino, de púrpura, de seda, de grana; toda madera olorosa, todo objeto de marfil, y todo objeto de madera preciosa, de bronce, de hierro, de mármol," 

\textbf{Rev 18:13} cinamomo y aromas, mirra e incienso, vino, aceite, flor de harina, trigo, bestias de carga, ovejas, caballos y coches, esclavos y almas de hombres. 

\textbf{Rev 18:14} Los frutos sabrosos a tu apetito te han faltado y todas las cosas más exquisitas y delicadas perecieron para ti y ya no serán halladas jamás. 

\textbf{Rev 18:15} Los mercaderes de estas cosas, que se enriquecían con ella, se detienen a lo lejos por el temor de su tormento, llorando y lamentándose, diciendo: 

\textbf{Rev 18:16} ¡Ay, ay de la ciudad grande, que se vestía de lino, púrpura y grana, y se adornaba de oro, piedras preciosas y perlas, porque en una hora quedó devastada tanta riqueza! 

\textbf{Rev 18:17} Todo piloto y navegante, los marineros y cuantos bregan en el mar, se detuvieron a lo lejos 

\textbf{Rev 18:18} y clamaron al contemplar el humo de su incendio y dijeron: ¿Quién había semejante a la ciudad grande? 

\textbf{Rev 18:19} Y arrojaron ceniza sobre sus cabezas, y gritaron, llorando y lamentándose, y diciendo: ¡Ay, ay de la ciudad grande, en la cual se enriquecieron todos cuantos tenían navíos en el mar, a causa de su suntuosidad, porque en una hora quedó devastada! 

\textbf{Rev 18:20} Regocíjate por ello, ¡oh cielo! y los santos y los apóstoles y los profetas, porque Dios ha juzgado nuestra causa contra ella. 

\textbf{Rev 18:21} Un ángel poderoso levantó una piedra, corno una rueda grande de molino, y la arrojó al mar, diciendo: Con tal ímpetu será arrojada Babilonia, la gran ciudad, y no será hallada. 

\textbf{Rev 18:22} Nunca más se oirá en ella la voz de los citaristas, de los músicos, de los flautistas y de los trompeteros, ni artesanos de ningún arte será hallado jamás en ti, y la voz de la muela no se oirá ya más en ti, 

\textbf{Rev 18:23} la luz de lámpara no lucirá más en ti, ni se oirá más la voz del esposo y de la esposa, porque tus comerciantes eran magnates de la tierra, porque con tus maleficios se han extraviado todas las naciones, 

\textbf{Rev 18:24} y en ella se halló la sangre de los profetas, y de los santos, y de todos los degollados sobre la tierra. 

\textbf{Rev 19:1} Después de esto oí una fuerte voz, como de una muchedumbre numerosa en el cielo, que decía: Aleluya, salud, gloria, honor y poder a nuestro Dios, 

\textbf{Rev 19:2} porque verdaderos y justos son sus juicios, pues ha juzgado a la gran ramera, que corrompía la tierra con su fornicación, y en ella ha vengado la sangre de sus siervos. 

\textbf{Rev 19:3} Y por segunda vez dijeron: Aleluya. El humo de la ciudad sube por los siglos de los siglos. 

\textbf{Rev 19:4} Cayeron de hinojos los veinticuatro ancianos y los cuatro vivientes, y adoraron a Dios, que está sentado en el trono, diciendo: Amén, aleluya. 

\textbf{Rev 19:5} Del trono salió una voz, que decía: Alabad a nuestro Dios todos sus siervos, y cuantos le teméis, pequeños y grandes. 

\textbf{Rev 19:6} Oí una voz como de gran muchedumbre, y como voz de muchas aguas, y como voz de fuertes truenos, que decía: Aleluya, porque ha establecido su reino el Señor, Dios todopoderoso;" 

\textbf{Rev 19:7} alegrémonos y regocijémonos, démosle gloria, porque han llegado las bodas del Cordero, y su esposa está dispuesta, 

\textbf{Rev 19:8} y fuele otorgado vestirse de lino brillante, puro,* pues el lino son las obras justas de los santos. 

\textbf{Rev 19:9} Y me dijo: Escribe: Bienaventurados los invitados al banquete de bodas del Cordero. Y me dijo: Estas son las palabras verdaderas de Dios. 

\textbf{Rev 19:10} Me arrojé a sus pies para adorarle, y me dijo: Mira, no hagas eso; consiervo tuyo soy y de tus hermanos, los que tienen el testimonio de Jesús. Adora a Dios. Porque el testimonio de Jesús es el espíritu de profecía." 

\textbf{Rev 19:11} Vi el cielo cubierto, y he aquí un caballo blanco, y el que lo montaba es llamado Fiel, Verídico, y con justicia juzga y hace la guerra. 

\textbf{Rev 19:12} Sus ojos son como llama de fuego, lleva en su cabeza muchas diademas, y tiene un nombre escrito que nadie conoce sino él mismo, 

\textbf{Rev 19:13} y viste un manto empapado en sangre, y tiene por nombre Verbo de Dios. 

\textbf{Rev 19:14} Le siguen los ejércitos celestes sobre caballos blancos, vestidos de lino blanco, puro. 

\textbf{Rev 19:15} De su boca sale una espada aguda para herir con ella a las naciones y El las regirá con vara de hierro, y El pisa el lagar del vino del furor de la cólera de Dios todopoderoso. 

\textbf{Rev 19:16} Tiene sobre su manto y sobre su muslo escrito su nombre: Rey de reyes, Señor de señores. 

\textbf{Rev 19:17} Vi un ángel puesto de pie en el sol, que gritó con una gran voz, diciendo a todas las aves que vuelan por lo alto del cielo: Venid, congregaos al gran festín de Dios, 

\textbf{Rev 19:18} para comer las carnes de los reyes, las carnes de los tribunos, las carnes de los valientes, las carnes de los caballos y de los que cabalgan en ellos, las carnes de todos los libres y de los esclavos, de los pequeños y de los grandes. 

\textbf{Rev 19:19} Y vi a la bestia, y a los reyes de la tierra, y a sus ejércitos, reunidos para hacer la guerra al que montaba el caballo y a su ejército. 

\textbf{Rev 19:20} Y fue aprisionada la bestia, y con ella el falso profeta, que hacía señales delante de ella, con las cuales extraviaba a los que habían recibido el carácter de la bestia y a los que adoraban su imagen; vivos fueron arrojados ambos al lago de fuego que arde con azufre." 

\textbf{Rev 19:21} Los demás fueron muertos por la espada que le salía de la boca al que montaba el caballo, y todas las aves se hartaron de sus carnes. 

\textbf{Rev 20:1} Vi un ángel que descendía del cielo, trayendo la llave del abismo y una gran cadena en su mano. 

\textbf{Rev 20:2} Cogió al dragón, la serpiente antigua, que es el diablo, Satanás, y le encadenó por mil años. 

\textbf{Rev 20:3} Le arrojó al abismo y cerró, y encima de él puso un sello para que no extraviase más a las naciones hasta terminados los mil años, después de los cuales será soltado por poco tiempo. 

\textbf{Rev 20:4} Vi tronos, y sentáronse en ellos, y fueles dado el poder de juzgar, y vi las almas de los que habían sido degollados por el testimonio de Jesús y por la palabra de Dios, y cuantos no habían adorado a la bestia, ni a su imagen, y no habían recibido la marca sobre su frente y sobre su mano; y vivieron y reinaron con Cristo mil años." 

\textbf{Rev 20:5} Los restantes muertos no vivieron hasta terminados los mil años. Esta es la primera resurrección. 

\textbf{Rev 20:6} Bienaventurado y santo el que tiene parte en la primera resurrección; sobre ellos no tendrá poder la segunda muerte, sino que serán sacerdotes de Dios y de Cristo y reinarán con El por mil años." 

\textbf{Rev 20:7} Cuando se hubieren acabado los mil años, será Satanás soltado de su prisión 

\textbf{Rev 20:8} y saldrá a extraviar a las naciones que moran en los cuatro ángulos de la tierra, a Gog y a Magog, y reunirlos para la guerra, cuyo ejército será como las arenas del mar. 

\textbf{Rev 20:9} Subirán sobre la anchura de la tierra, y cercarán el campamento de los santos y la ciudad amada. Pero descenderá fuego del cielo y los devorará. 

\textbf{Rev 20:10} El diablo, que los extraviaba, será arrojado en el estanque de fuego y azufre, donde están también la bestia y el falso profeta, y serán atormentados día y noche por los siglos de los siglos. 

\textbf{Rev 20:11} Vi un trono alto y blanco, y al que en él se sentaba, de cuya presencia huyeron el cielo y la tierra, y no dejaron rastro de sí. 

\textbf{Rev 20:12} Vi a los muertos, grandes y pequeños, que estaban delante del trono; y fueron abiertos los libros, y fue abierto otro libro, que es el libro de la vida. Fueron juzgados los muertos, según sus obras, según las obras que estaban escritas en los libros." 

\textbf{Rev 20:13} Entregó el mar los muertos que tenía en su seno, y asimismo la muerte y el infierno entregaron los que tenían, y fueron juzgados cada uno según sus obras. 

\textbf{Rev 20:14} La muerte y el infierno fueron arrojados al estanque de fuego; ésta es la segunda muerte, el estanque de fuego," 

\textbf{Rev 20:15} y todo el que no fue hallado escrito en el libro de la vida fue arrojado en el estanque de fuego. 

\textbf{Rev 21:1} Vi un cielo nuevo y una tierra nueva, porque el primer cielo y la primera tierra habían desaparecido; y el mar no existía ya." 

\textbf{Rev 21:2} Y vi la ciudad santa, la nueva Jerusalén, que descendía del cielo, del lado de Dios, ataviada como una esposa que se engalana para su esposo. 

\textbf{Rev 21:3} Oí una voz grande, que del trono decía: He aquí el tabernáculo de Dios entre los hombres, y erigirá su tabernáculo entre ellos, y ellos serán su pueblo y el mismo Dios será con ellos, 

\textbf{Rev 21:4} y enjugará las lágrimas de sus ojos, y la muerte no existirá más, ni habrá duelo, ni gritos, ni trabajo, porque todo esto es ya pasado. 

\textbf{Rev 21:5} Y dijo el que estaba sentado en el trono: He aquí que hago nuevas todas las cosas. Y dijo: Escribe, porque éstas son las palabras fieles y verdaderas. 

\textbf{Rev 21:6} Díjome: Hecho está. Yo soy el alfa y la omega, el principio y el fin. Al que tenga sed le daré gratis de la fuente de agua de vida. 

\textbf{Rev 21:7} El que venciere heredará estas cosas, y seré su Dios, y él será mi hijo. 

\textbf{Rev 21:8} Los cobardes, los infieles, los abominables, los homicidas, los fornicadores, los hechiceros, los idólatras y todos los embusteros tendrán su parte en el estanque que arde con fuego y azufre, que es la segunda muerte. 

\textbf{Rev 21:9} Vino uno de los siete ángeles que tenían las siete copas, llenas de las siete últimas plagas, y habló conmigo y me dijo: Ven y te mostraré la novia, la esposa del Cordero. 

\textbf{Rev 21:10} Me llevó en espíritu a un monte grande y alto, y me mostró la ciudad santa, Jerusalén, que descendía del cielo, de parte de Dios, que tenía la gloria de Dios. 

\textbf{Rev 21:11} Su brillo era semejante a la piedra más preciosa, como la piedra de jaspe pulimentada. 

\textbf{Rev 21:12} Tenía un muro grande y alto y doce puertas, y sobre las doce puertas doce ángeles y nombres escritos, que son los nombres de las doce tribus de los hijos de Israel: 

\textbf{Rev 21:13} de la parte de oriente, tres puertas; de la parte del norte, tres puertas; de la parte del mediodía, tres puertas, y de la parte del poniente, tres puertas." 

\textbf{Rev 21:14} El muro de la ciudad tenía doce hiladas, y sobre ellas los nombres de los doce apóstoles del Cordero. 

\textbf{Rev 21:15} El que hablaba conmigo tenía una medida, una caña de oro, para medir la ciudad, sus puertas y su muro. 

\textbf{Rev 21:16} La ciudad estaba asentada sobre una base cuadrangular, y su longitud era tanta como su anchura. Midió con la caña la ciudad, y tenía doce mil estadios, siendo iguales su longitud, su latitud y su altura. 

\textbf{Rev 21:17} Midió su muro, que tenía ciento cuarenta y cuatro codos, medida humana, que era la del ángel. 

\textbf{Rev 21:18} Su muro era de jaspe, y la ciudad oro puro, semejante al vidrio puro;" 

\textbf{Rev 21:19} y las hiladas del muro de la ciudad eran de todo género de piedras preciosas: la primera, de jaspe; la segunda, de zafiro; la tercera, de calcedonia; la cuarta, de esmeralda;" 

\textbf{Rev 21:20} la quinta, de sardónica; la sexta, de cornalina; la séptima, de crisólito; la octava, de berilo; la novena, de topacio; la décima, de crisoprasa; la undécima, de jacinto, y la duodécima, de amatista." 

\textbf{Rev 21:21} Las doce puertas eran doce perlas, cada una de las puertas era una perla, y la plaza de la ciudad era de oro puro, como vidrio transparente. 

\textbf{Rev 21:22} Pero templo no vi en ella, pues el Señor, Dios todopoderoso, con el Cordero, era su templo. 

\textbf{Rev 21:23} La ciudad no había menester de sol ni de luna que la iluminasen, porque la gloria de Dios la iluminaba y su lumbrera era el Cordero. 

\textbf{Rev 21:24} A su luz caminarán las naciones, y los reyes de la tierra llevarán a ella su gloria. 

\textbf{Rev 21:25} Sus puertas no se cerrarán de día, pues noche allí no habrá, 

\textbf{Rev 21:26} y llevarán a ella la gloria y el honor de las naciones. 

\textbf{Rev 21:27} En ella no entrará cosa impura ni quien cometa abominación y mentira, sino los que están escritos en el libro de la vida del Cordero. 

\textbf{Rev 22:1} Y me mostró un río de agua de vida, clara como el cristal, que salía del trono de Dios y del Cordero. 

\textbf{Rev 22:2} En medio de la calle y a un lado y otro del río había un árbol de vida que daba doce frutos, cada fruto en su mes, y las hojas del árbol eran saludables para las naciones. 

\textbf{Rev 22:3} No habrá ya maldición alguna, y el trono de Dios y del Cordero estará en ella, 

\textbf{Rev 22:4} y sus siervos le servirán, y verán su rostro, y llevarán su nombre sobre la frente. 

\textbf{Rev 22:5} No habrá ya noche, ni tendrá necesidad de luz de antorcha, ni de luz del sol, porque el Señor Dios los alumbrará, y reinarán por los siglos de los siglos. 

\textbf{Rev 22:6} Y me dijo: Estas son las palabras fieles y verdaderas, y el Señor, Dios de los espíritus de los profetas, envió su ángel para mostrar a sus siervos las cosas que están para suceder pronto. 

\textbf{Rev 22:7} He aquí que vengo presto. Bienaventurado el que guarda las palabras de la profecía de este libro. 

\textbf{Rev 22:8} Y yo, Juan, oí y vi estas cosas. Cuando las oí y vi, caí de hinojos para postrarme a los pies del ángel que me las mostraba. 

\textbf{Rev 22:9} Pero me dijo: No hagas eso, pues soy consiervo tuyo, y de tus hermanos los profetas, y de los que guardan las palabras de este libro; adora a Dios." 

\textbf{Rev 22:10} Y me dijo: No selles los discursos de la profecía de este libro, porque el tiempo está cercano. 

\textbf{Rev 22:11} El que es injusto continúe aún en sus injusticias, el torpe prosiga en sus torpezas, el justo practique aún la justicia y el santo santifíquese más. 

\textbf{Rev 22:12} He aquí que vengo presto, y conmigo mi recompensa, para dar a cada uno según sus obras. 

\textbf{Rev 22:13} Yo soy el alfa y la omega, el primero y el último, el principio y el fin. 

\textbf{Rev 22:14} Bienaventurados los que lavan sus túnicas para tener derecho al árbol de la vida y a entrar por las puertas que dan acceso a la ciudad. 

\textbf{Rev 22:15} Fuera perros, hechiceros, fornicarios, homicidas, idólatras y todos los que aman y practican la mentira. 

\textbf{Rev 22:16} Yo, Jesús, envié a un ángel para testificaros estas cosas sobre las iglesias. Yo soy la raíz y el linaje de David, la estrella brillante de la mañana. 

\textbf{Rev 22:17} Y el Espíritu y la Esposa dicen: Ven. Y el que escucha diga: Ven. Y el que tenga sed, venga, y el que quiera tome gratis el agua de la vida. 

\textbf{Rev 22:18} Yo atestiguo a todo el que escucha mis palabras de la profecía de este libro que, si alguno añade a estas cosas, Dios añadirá sobre él las plagas escritas en este libro;" 

\textbf{Rev 22:19} y si alguno quita de las palabras del libro de esta profecía, quitará Dios su parte del árbol de la vida y de la ciudad santa que están escritos en este libro. 

\textbf{Rev 22:20} Dice el que testifica estas cosas: Sí, vengo pronto. Amén. Ven, Señor Jesús. 

\textbf{Rev 22:21} La gracia del Señor Jesús sea con todos* Amén. 

\nopagebreak

\blEndBook




%\end{verseindent}

\end{document}

